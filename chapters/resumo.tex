Os microbianos patogénicos são atualmente responsáveis por mais de 400 milhões de anos de vida perdidos anualmente em todo o mundo, uma carga maior do que o cancro ou as doenças cardiovasculares. Além do surgimento de patogéneos virulentos, o aumento da resistência a antimicrobianos representa uma das maiores ameaças à saúde humana. A microbiologia clínica é uma disciplina focada na caracterização rápida de amostras para a identificação de agentes causadores de doença (diagnóstico) e monitorizar a epidemiologia de doenças infeciosas (saúde pública), incluindo a deteção de surtos e prevenção de infeções. 

Em microbiologia clinica, tanto no ramo de diagnóstico clinico como no ramo de prevenção e vigilância, a identificação rápida de um possível agente de infeção é um passo essencial. 
A metagenómica, definida como a sequenciação e análise genómica de uma dada comunidade de microrganismos de um determinado ambiente por técnicas independentes de cultura, apresenta-se como uma alternativa válida à deteção de possíveis patogénios sem ser necessário conhecimento \textit{a priori} do possível causador de doença.

O trabalho aqui apresentado tem como objetivo avaliar a implementação de métodos de metagenómica em microbiologia clínica como alternativa aos métodos padrão até então implementados. Tem como especial foco a análise bioinformática deste tipo de dados que, dado ao seu volume e características, requer um processamento dedicado e delicado particularmente se implementado em diagnostico. Para efeitos deste trabalho, as infeções bacterianas e virais serão o principal foco de interesse.  

A Sequenciação é um processo que determina a ordem de nucleótidos numa molécula especifica de DNA de qualquer organismo. Esta informação é muito útil na investigação e na prática clínica, dado que permite compreender qual o tipo de informação genética que as moléculas carregam. 

O ADN possui uma estrutura de dupla hélice que é composta por quatro bases que emparelham sempre com o mesmo par. O nucleótido Adenina (A) emparelha com o nucleótido Timina (T) e o nucleótido Citosina (C) com o nucleótido Guanina (G). Estes pares são a base da molécula de DNA e a base da replicação, divisão celular e dos métodos de sequenciação. O conhecimento das sequências de DNA e RNA nos últimos anos tem-se tornado indispensável. Atualmente existem já bem estabelecidas três tipos de sequenciação de DNA, a de baixo rendimento ou 1ª geração, a de alto rendimento ou 2ª geração e a sequenciação de cadeias longas ou 3ª geração.

A sequenciação total do genoma é atualmente parte da rotina laboratorial aquando da tipagem de microrganismos dado o seu elevado poder discriminatório, tendo-se tornado numa ferramenta importante em saúde publica, permitindo a identificação de patogéneos, identificação do estabelecimento rotas de transmissão e controlo de surtos. Contudo, a implementação da sequenciação total do genoma em diagnósticos exige várias adaptações no fluxo de trabalho laboratorialmente, desde o trabalho de bancada (extração, preparação da biblioteca, sequenciamento), até à analise bioinformática onde os dados de genómica são analisados e seus resultados interpretados por pessoal especializado.

Os métodos de sequenciação de ADN estão a ser cada vez mais adotados em microbiologia clínica, contudo requerem um conhecimento \textit{a priori} do que uma amostra clínica ou paciente irá conter. Uma das possibilidades para ultrapassar esta limitação é através da implementação da metagenómica. Enquanto a maioria dos ensaios moleculares visa apenas um número limitado de agentes patogénicos, a abordagem metagenómica caracteriza todo o ADN ou ARN presente numa amostra, permitindo analisar de todo o microbioma. Se pode ou não substituir totalmente a microbiologia rotineira depende de várias condições e desenvolvimentos futuros, tanto tecnológicos como computacionais. A análise da bioinformática, exigida devido à quantidade de dados produzidos pelas tecnologias genómicas de sequenciação, representa uma das principais limitações para a aplicabilidade desta metodologia em microbiologia clinica.

Neste trabalho, a metagenómica foi aplicada com sucesso a nove amostras de fluido corporal e uma amostra de tecido de pacientes do University Medical Center Groningen, com diferentes graus de contaminação por ADN humano, e comparada com métodos microbiológicos baseados em cultura, de modo a comparar e a avaliar a precisão e fiabilidade das análises bioinformáticas efetuadas. Além disso, esta metodologia foi aplicada a oito amostras de água concentradas também recolhidas do University Medical Center Groningen. Numa das amostras, é relatada a deteção de uma nova variante de um gene a resistência a antimicrobianos \textit{MCR-5}, denominado de \textit{MCR-5.4}.

Com as lições aprendidas com o processamento das amostras clínicas e ambientais recolhidas, desenvolvemos DEN-IM, uma workflow fácil de utiliza e reproduzível, para a análise de dados de sequenciação de 2ª geração de Dengue, tanto a partir de abordagens de amplificação de fragmentos do genoma total como de metagenómia. A sequenciação genómica é a abordagem mais informativa para monitorizar da disseminação viral e da diversidade genética, proporcionando, num único passo metodológico, a identificação e caracterização de todo o genoma viral ao nível do nucleótido. O DEN-IM foi concebido para realizar uma análise abrangente, com o objetivo de gerar sequências completas de DENV e identificar o seu serótipo e genótipo. O DEN-IM pode também desempenhar um papel na identificação de casos de coinfecção por duas ou mais estirpes de DENV cuja prevalência é cada vez mais prevalente em áreas endémicas. 

A metagenómica pode oferecer uma deteção microbiana abrangente e caracterização de amostras clínicas complexas. A montagem de dados brutos de sequenciação em sequências mais longas, que oferecem informação contextual e uma imagem mais completa da comunidade microbiana em questão, é um passo fundamental na análise de dados de metagenómica. Contudo, este processo de montagem é frequentemente um passo limitante na obtenção de resultados confiáveis e reprodutíveis. Assim foi desenvolvido o LMAS, um workflow automatizado que permite aos utilizadores avaliar o desempenho de software de montagem de genomas, tanto tradicionais como específicos para dados de metagenómica, dado uma comunidade de composição conhecida. À semelhança do DEN-IM, a implementação do LMAS garante a transparência e reprodutibilidade dos resultados obtidos, apresentados num relatório interativo HTML onde podem ser exploradas métricas de desempenho globais e específicas da referência utilizada. 

Apesar da relativa normalização do processo para adquirir dados genómicos, seja através de toda a sequenciação do genoma ou da metagenómica, há uma miríade de diferentes ferramentas disponíveis para executar na deteção \textit{in silico} de resistência antimicrobiana. Alguns usam a sua própria base de dados com sequências de referência e cada uma gera um relatório único e não padronizado dos genes ou variantes que podem possivelmente conferir resistência numa determinada amostra. Esta é uma enorme barreira para a comparação dos resultados obtidos. Dada esta situação, é apresentada uma especificação para a detecção de genes ou variantes que conferem uma resistência antimicrobiana, disponibilizada no software hAMRonization, capaz de agregar resultados de uma grande variedade de ferramentas de deteção de resistência a antimicrobianos, tanto agnósticas como específicas de uma determinada espécie, fornecendo um relatório unificado, em forma tabular, JSON ou através de um ficheiro HTML interativo. 

Apesar da riqueza da informação genómica disponível, o mesmo não se observa para a informação contextual que a acompanha. Seguindo a mesma abordagem desenvolvida para a deteção de resistência a antimicrobianos, foi concebida uma especificação desta vez aplicada aos dados contextuais SARS-CoV-2 baseados em normas desenvolvidas pela comunidade e publicamente disponíveis. Esta especificação está implementada e disponível através de um modelo de recolha de dados, bem como uma variedade de protocolos e ferramentas para apoiar a harmonização e submissão de dados de sequenciação e informação contextual para bio-repositórios públicos. 

Ao longo deste trabalho, tornou-se óbvio que os algoritmos computacionais se tornaram num componente essencial da investigação do microbioma, com grandes esforços da comunidade científica para elevar os padrões de desenvolvimento e distribuição de código. Apesar destes esforços, a sustentabilidade e a reprodutibilidade são questões importantes, uma vez que a continuação da validação através de testes de software ainda não é uma prática amplamente adotada. Num esforço para manter boas práticas de engenharia de software, reportamos sete recomendações que ajudam investigadores a implementar testes de software em bioinformática. Propomos testes de software colaborativos como uma oportunidade para envolver continuamente quem desenvolve software, assim como quem o utiliza no quotidiano, unificando o trabalho científico em todos os domínios. Como os testes de software automatizados permanecem em subuso no desenvolvimento de software científico, o nosso conjunto de recomendações não só garante que o esforço adequado pode ser investido na produção de software de alta qualidade e robusto, mas também aumenta o envolvimento da comunidade na sua sustentabilidade. 

O impacto e aplicabilidade da metagenómica na microbiologia clínica, incluindo o diagnóstico e a vigilância e prevenção de infeções, foi avaliado, com os desafios únicos de ambos destacados. Uma forte aposta na normalização e reprodutibilidade dos resultados obtidos, com o uso de novas tecnologias é da absoluta necessidade para aplicação sustentável as soluções de análise de dados para este tipo de dados. Transparência, escalabilidade e facilidade de instalação são essenciais, independentemente das ferramentas escolhidas. As soluções adotadas ao longo deste trabalho, aliadas a documentação clara e fácil de seguir, visam reduzir a barreira de entrada ao realizar análises detalhadas que sejam complexas e computacionalmente caras na sua natureza. Adicionalmente, a produção de relatórios intuitivos, responsivos e fáceis de seguir, permite o resumo dos resultados-chave, bem como a sua exploração detalhada por parte das partes interessadas, seja pessoal bioinformático ou peritos na área de especialização. Isto representa o contributo mais importante para a redução da barreira entre quem produz os dados e quem tem a capacidade de tomar decisões informadas com base nestes. 