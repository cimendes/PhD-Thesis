Os patogéneos microbianos são actualmente responsáveis por mais de 400 milhões de anos de vida perdidos anualmente em todo o mundo, uma carga maior do que o cancro ou as doenças cardiovasculares. Além do surgimento de patogéneos virulentos, o aumento da resistência a antimicrobianos representa uma das maiores ameaças à saúde humana. A microbiologia clínica é uma disciplina focada na caracterização rápida de amostras para a idenficiação de agentes causadores de doença (diagnóstico) e monitorizar a epidemiologia de doenças infecciosas (saúde pública), incluindo a detecção de surtos e prevenção de infecções. 

Em microbiologia clinica, tanto no ramo de diagnóstico clinico como no ramo de prevenção e vigilância de infecções, a identificação rápida de um possível agente de infecção é um passo essencial. 
A metagenómica, definida como a sequênciação e análise genómica de uma dada comunidade de microrganismos de um determinado ambiente por técnicas independentes de cultura ou de métodos moleculares, como reação em cadeia da polimerase, apresenta-se como uma alternativa válida à detecção de possiveis patogénios sem ser necessário conhecimento \textit{a priori} do possivel causador de doença.

O trabalho aqui apresentado tem como objectivo avaliar a implementação de métodos de metagenómica em microbiologia clínica como alternativa aos métodos padrão até então implementados. Tem como especial foco a análise bioinformática deste tipo de dados que, dado ao seu volume e características, requer um processamento dedicato e delicado particularmente se implementado em diagnostico. Para efeitos deste trabalho, as infecções bacterianas e virais serão o principal foco de interesse.  

A Sequenciação é um processo que determina a ordem de nucleótidos numa molécula especifica de DNA de qualquer organismo. Esta informação é muito útil na investigação e na prática clínica, dado que permite compreender qual o tipo de informação genética que as moléculas carregam. 

O DNA possui uma estrutura de dupla hélice que é composta por quatro bases que emparelham sempre com o mesmo par. O nucleótido Adenina (A) emparelha com o nucleótido Timina (T) e o nucleótido Citosina (C) com o nucleótido Guanina (G). Estes pares são a base da molécula de DNA e a base da replicação, divisão celular e dos métodos de sequenciação. O conhecimento das sequências de DNA e RNA nos últimos anos tem-se tornado indispensável. Atualmente existem já bem estabelecidas três tipos de sequenciação de DNA, a de baixo rendimento ou 1ª geração, a de alto rendimento ou 2ª geração e a sequenciação de cadeias longas ou 3ª geração.

A sequenciação total do genoma é actualmente parte da rotina laboratorial aquando da tipagen de microorganismos dado o seu elevado poder discriminatório, tendo-se tornado numa ferramenta particularmente na vigilância e prevenção de infecções, permitindo a identificação  de patogéneos, identificação do estabelecimento rotas de transmissão e controlo de surtos. Contudo, a implementação da sequenciação total do genoma em diagnósticos exige várias adaptações no fluxo de trabalho laboratorialmente, desde o trabalho de bancada (extração, preparação da biblioteca, sequenciamento), até à analise bioinformática onde os dados de genómica são analisados e seus resultados interpretados por pessoal especializado.