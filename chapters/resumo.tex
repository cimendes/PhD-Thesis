A microbiologia é a ciência que estuda os microrganismos, um vasto e diverso grupo de organismos microscópicos que podem ser encontrados como células undividualizadas ou em agrupamentos celulares. Dentro dos microrganismos estão também incluídos os vírus que, apesar de terem uma estrutura acelular e sendo sempre encontrados associados a outros seres vivos, são capazes de realizar os seus próprios processos vitais de crescimento. Todas as células são compostas por quatro componentes químicos (biomoléculas): proteínas, ácidos nucleicos, lípidos e polissacáridos. A célula corresponde à unidade fundamental do ser vivo, sendo os microrganismos representados por uma grande diversidade de organismos com diferentes características celulares. Devido a esta diversidade, a microbiologia encontra-se dividida em diferentes áreas: a virologia, a bacteriologia, a micologia e a parasitologia. Para efeitos desta dissertação apenas nos iremos focar em duas áreas: a virologia, ciência que se dedica ao estudo dos vírus e das viroses, e a bacteriologia, ciência que investiga o que é relativo às bactérias, nome geral dado aos microrganismos unicelulares de formas variadas, sem membrana nuclear. 

No ser humano, o maior número de bactérias está na microbiota intestinal, seguido pela pele. A grande maioria das bactérias são inofensivas devido aos efeitos protectores do sistema imunitário, e muitas são benéficas, particularmente na microbiota intestinal. No entanto, várias espécies têm potencial patogénico e causam doenças. Os microbianos patogénicos são atualmente responsáveis por mais de 400 milhões de anos de vida perdidos anualmente em todo o mundo, uma carga maior do que o cancro ou as doenças cardiovasculares. Além do aparecimento de novos agentes patogénicos ou de uma maior virulência dos agentes conhecidos, o aumento da resistência aos fármacos antimicrobianos representa uma das maiores ameaças à saúde humana. A microbiologia clínica é então uma disciplina focada na caracterização rápida de amostras para a identificação de agentes causadores de doença (diagnóstico) e monitorizar a epidemiologia das doenças infeciosas (saúde pública), incluindo a deteção de surtos e prevenção das infeções. 

Em microbiologia clínica, tanto na vertente de diagnóstico como na vertente de saúde pública, a identificação rápida de um possível agente de infeção é um passo essencial. A identificação é tipicamente efetuada pela caracterização bioquímica e fenotípica dos microrganismos, como por exemplo pelo crescimento em meios de cultura adequados seguido da análise de múltiplas características fisiológicas, metabólicas e bioquímicas. Com o advento das técnicas de biologia molecular e imuno-serologia, a microbiologia sofreu uma revolução metodológica, permitindo a identificação rápida e específica das espécies microbianas. Contudo, é necessário o conhecimento \textit{a priori} do possível causador de doença, muitas vezes por cultura. A metagenómica, definida como a sequenciação e análise genómica de uma comunidade de microrganismos, de um determinado ambiente, por técnicas independentes de cultura, apresenta-se como uma alternativa válida à deteção de possíveis agentes patogénicos num único passo metodológico. A bioinformática, definida como a área computacional da microbiologia molecular, é composta por tarefas especializadas onde análises de dados são utilizadas para processamento e análise \textit{in silico} de informação sobre biomoléculas, em oposição aos métodos \textit{in vivo} (num organismo vivo) ou \textit{in vitro} (num ambiente artificial) tradicionalmente utilizados. É composta pelo desenvolvimento de \textit{software} ou de \textit{workflows}, onde software é encadeado para o processamento automático de informação, e pela interpretação dos resultados obtidos.

O trabalho aqui apresentado tem como objetivo avaliar a implementação de métodos de metagenómica em microbiologia clínica, como alternativa aos métodos padrão ainda hoje implementados. Tem como especial foco a análise bioinformática deste tipo de dados que, dado ao seu volume e características, requer um processamento dedicado e delicado, particularmente se implementado em diagnóstico. 

Os ácidos nucleicos são compostos por dois grupos de moléculas: o ácido desoxirribonucleico (ADN) e o ácido ribonucleico (ARN). O ADN possui uma estrutura de dupla hélice que é composta por quatro bases que emparelham sempre com o mesmo par. O nucleótido Adenina (A) emparelha com o nucleótido Timina (T) e o nucleótido Citosina (C) com o nucleótido Guanina (G). Estes pares são a base da molécula de ADN e a da replicação, da divisão celular, bem como dos métodos de sequenciação. O ácido ribonucleico (ARN) é também formado por uma cadeia de nucleótidos, mas diferentemente do ADN, sendo monocatenários, embora possam dobrar-se sobre si mesmos. Têm uma composição semelhante, com a diferença em que o nucleótido T é substituído pelo nucleótido Uracilo (U). O dogma principal da biologia dita que a informação genética é passada do ADN para o ARN através de um processo de transcrição, sendo o último expresso em proteínas que acatam uma certa função através da sua tradução. 

A sequenciação é um processo que determina a ordem de nucleótidos numa molécula específica de ADN de qualquer organismo. Esta informação é muito útil na investigação e na prática clínica, dado que permite compreender qual o tipo de informação genética que as moléculas carregam. As moléculas de ARN também são possíveis de serem sequenciadas após a sua conversão por transcrição revertida em ADN. O conhecimento das sequências de ADN e ARN nos últimos anos tem-se tornado indispensável. Atualmente, existem já bem estabelecidos três tipos de sequenciação de ADN, a de baixo rendimento ou 1ª geração, a de alto rendimento ou 2ª geração e a sequenciação de cadeias longas ou 3ª geração.

A sequenciação total do genoma é, atualmente, parte da rotina laboratorial aquando da tipagem de microrganismos. Dado o seu elevado poder discriminatório, tornou-se numa ferramenta importante em saúde publica. Permite a identificação de agentes patogénicos, identificação e estudo de vias de transmissão informando as medidas de controlo de surtos. Contudo, a implementação da sequenciação total do genoma em diagnóstico exige várias adaptações no fluxo de trabalho, a nível laboratorial; desde o trabalho de bancada (extração, preparação da biblioteca, sequenciamento), até à analise bioinformática, onde os dados de genómica são analisados e os seus resultados interpretados por pessoal especializado.

Os métodos de sequenciação de ADN estão a ser cada vez mais adotados em microbiologia clínica, contudo, tal como os métodos moleculares, requerem um conhecimento \textit{a priori} de quais os microrganismos que a amostra de um doente poderá conter. Uma das possibilidades para ultrapassar esta limitação é através da implementação da metagenómica. Enquanto a maioria dos ensaios moleculares visa apenas um número limitado de agentes patogénicos, a abordagem metagenómica caracteriza todo o ADN ou ARN presente numa amostra, permitindo identificar e analisar todos os microrganismos presentes. Se poderá ou não substituir totalmente as metodologias atualmente implementadas na rotina de microbiologia clínica, depende de várias condições e desenvolvimentos futuros, tanto tecnológicos como computacionais. A análise bioinformática, exigida devido à quantidade de dados produzidos pelas tecnologias genómicas de sequenciação, representa uma das principais limitações para a aplicabilidade desta metodologia.

Neste trabalho, a metagenómica foi aplicada com sucesso a nove amostras de diferentes doentes do University Medical Center Groningen, com diferentes graus de contaminação por ADN humano, e comparada com métodos microbiológicos baseados em cultura, de modo a avaliar a precisão e fiabilidade das análises bioinformáticas efetuadas. Além disso, esta metodologia foi aplicada a oito amostras de água também recolhidas no University Medical Center Groningen. Numa das amostras, é relatada a deteção de uma nova variante de um gene a resistência a antimicrobianos \textit{MCR-5}, denominado de \textit{MCR-5.4}.

Com as lições aprendidas com o processamento das amostras clínicas e ambientais recolhidas, desenvolvemos DEN-IM, uma \textit{workflow} de fácil utilização e reproduzível, para a análise de dados de sequenciação de 2ª geração do vírus da Dengue (DENV), tanto a partir de abordagens de amplificação de fragmentos do genoma total, como de metagenómica. A sequenciação genómica é a abordagem mais informativa para monitorizar a disseminação viral e a diversidade genética, proporcionando, num único passo metodológico, a identificação e caracterização de todo o genoma viral. O DEN-IM foi concebido para realizar uma análise abrangente, com o objetivo de gerar sequências completas de DENV e identificar o seu serótipo e genótipo. O DEN-IM pode também desempenhar um papel na identificação de casos de coinfecção, por duas ou mais estirpes de DENV, cuja prevalência é cada vez mais elevada em áreas endémicas. 

A metagenómica pode oferecer uma deteção microbiana abrangente e caracterização de amostras clínicas complexas. A montagem de dados brutos de sequenciação em sequências mais longas, que oferecem informação contextual e uma imagem mais completa da comunidade microbiana em questão, é um passo fundamental na análise de dados de metagenómica. Contudo, este processo de montagem é frequentemente um passo limitante na obtenção de resultados confiáveis e reprodutíveis. Assim, foi desenvolvido o LMAS, uma \textit{workflow} automatizada que permite aos utilizadores avaliar o desempenho de software de montagem de genomas, tanto tradicionais como específicos para dados de metagenómica, dada uma comunidade de composição conhecida. À semelhança do DEN-IM, a implementação do LMAS garante a transparência e reprodutibilidade dos resultados obtidos, apresentados num relatório interativo HTML, onde podem ser exploradas métricas de desempenho globais e específicas da referência utilizada. 

Apesar da relativa normalização do processo para adquirir dados genómicos, seja através de toda a sequenciação do genoma ou da metagenómica, há uma miríade de diferentes ferramentas disponíveis para executar na deteção \textit{in silico} de resistência a fármacos antimicrobianos. Alguns usam a sua própria base de dados com sequências de referência e cada uma gera um relatório único e não padronizado dos genes ou variantes que podem possivelmente conferir resistência numa determinada amostra. Esta é uma enorme barreira para a comparação dos resultados obtidos. Dada esta situação, é apresentada uma especificação para a detecção de genes ou variantes que conferem resistência, disponibilizada no software hAMRonization, capaz de agregar resultados de uma grande variedade de ferramentas de deteção de resistência a antimicrobianos, tanto agnósticas como específicas de uma determinada espécie, fornecendo um relatório unificado, em forma tabular, JSON ou através de um ficheiro HTML interativo.

Apesar da riqueza da informação genómica disponível, o mesmo não se observa para a informação contextual que a acompanha. Seguindo a mesma abordagem desenvolvida para a deteção de resistência a antimicrobianos, foi concebida uma especificação, desta vez aplicada aos dados contextuais do coronavírus do síndrome respiratório agudo grave 2 (SARS-CoV-2), baseados em normas desenvolvidas pela comunidade e publicamente disponíveis. Esta especificação está implementada e disponível através de um modelo de recolha de dados, bem como numa variedade de protocolos e ferramentas para apoiar a harmonização e submissão de dados de sequenciação e informação contextual para bio-repositórios públicos.

Ao longo deste trabalho, tornou-se óbvio que os algoritmos computacionais são um componente essencial da investigação tanto do microbioma como da etiologia das mais diversas infecções. Têm havido grandes esforços por parte da comunidade científica para se elevar os padrões de desenvolvimento e distribuição de \textit{software} para este fim. Apesar destes esforços, a sustentabilidade e a reprodutibilidade continuam a ser questões importantes, uma vez que a continuação da validação através de testes de software ainda não é uma prática amplamente adotada. Num esforço para manter boas práticas de engenharia de \textit{software}, reportamos sete recomendações que ajudam investigadores a implementar testes de \textit{software} em bioinformática. Propomos a utilização destes testes como uma oportunidade para envolver quem desenvolve \textit{software}, assim como quem o utiliza no quotidiano, unificando o trabalho científico em todos os domínios. Como os testes de \textit{software} automatizados permanecem em subuso em \textit{software} científico, o nosso conjunto de recomendações não só garante que o esforço adequado pode ser investido na produção de \textit{software} de alta qualidade e robusto, mas também aumenta o envolvimento da comunidade na sua sustentabilidade.

O impacto e aplicabilidade da metagenómica na microbiologia clínica, incluindo o diagnóstico, a vigilância e prevenção de infeções, foi avaliado, com os desafios únicos de cada um destacados. Uma forte aposta na normalização e reprodutibilidade dos resultados obtidos, com o uso de novas tecnologias é absolutamente necessário para aplicação sustentável de soluções de análise de dados, bem como para a aplicabilidade da metagenómica em microbiologia clínica. A transparência, escalabilidade e facilidade de instalação são características essenciais no \textit{software} desenvolvido e utilizado neste tipo de análise, independentemente das ferramentas escolhidas. As soluções adotadas ao longo deste trabalho, aliadas a uma documentação clara e fácil de seguir, visam reduzir a barreira de entrada ao realizar análises detalhadas que sejam complexas e computacionalmente caras na sua natureza. Adicionalmente, a produção de relatórios intuitivos, interactivos e fáceis de seguir, permite o resumo dos resultados-chave, bem como a sua exploração detalhada pelas partes interessadas, seja pessoal bioinformático ou peritos na área de especialização. Isto representa o contributo mais importante para a redução da barreira entre quem produz os dados e quem tem a capacidade de tomar decisões informadas com base nestes. 

Com o trabalho aqui apresentado, foram identificadas características chave necessárias em \textit{workflows} e \textit{software} bioinformático que, em ultima instância, facilitam a implementação da metagenómica na rotina laboratorial tanto no diagnóstico como em saúde pública.
