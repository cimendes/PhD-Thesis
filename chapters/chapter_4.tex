\mbox{}\\
\vspace{8cm}

This chapter is a reproduction of the following publication:

C. I. Mendes, E. Lizarazo, M. P. Machado, D. N. Silva, A. Tami, M. Ramirez, N. Couto, J. W. A. Rossen, J. A. Carriço, DEN-IM: dengue virus genotyping from amplicon and shotgun metagenomic sequencing. Microbial Genomics, Volume 6, Issue 3, March 2020. DOI: \url{https://doi.org/10.1099/mgen.0.000328}

The supplementary information referred throughout the text can be consulted in this chapter before the section of references. 

LA LA LA 

\cleardoublepage 

\begin{center}
\large
\textbf{DEN-IM: dengue virus genotyping from amplicon and shotgun metagenomic sequencing}
\end{center}

Catarina I Mendes$^1,2,*$, 
Erley Lizarazo$^2,*$,
Miguel P Machado$^1$, 
Diogo N Silva$^1$,
Adiana Tami$^2$,
Mário Ramirez$^1$, 
Natacha Couto$^2$, 
John W A Rossen$^2$ and João A Carriço$^1$


$^1$Instituto de Microbiologia, Instituto de Medicina Molecular, Faculdade de Medicina, Universidade de Lisboa, Lisboa, Portugal 

$^2$University of Groningen, University Medical Center Groningen, Department of Medical Microbiology and Infection Prevention, Groningen, The Netherlands

$^*$Contributed equally

\section{Abstract}

Dengue virus (DENV) represents a public health and economic burden in affected countries. The availability of genomic data is key to understanding viral evolution and dynamics, supporting improved control strategies. Currently, the use of High Throughput Sequencing (HTS) technologies, which can be applied both directly to patient samples (shotgun metagenomics) and PCR amplified viral sequences (targeted metagenomics), is the most informative approach to monitor the viral dissemination and genetic diversity.

Despite many advantages, these technologies require bioinformatics expertise and appropriate infrastructure for the analysis and interpretation of the resulting data. In addition, the many software solutions available can hamper reproducibility and comparison of results.
Here we present DEN-IM, a one-stop, user-friendly, containerised and reproducible workflow for the analysis of DENV sequencing data, both from shotgun and targeted metagenomics approaches. It is able to infer the DENV coding sequence (CDS), identify the serotype and genotype, and generate a phylogenetic tree. It can easily be run on any UNIX-like system, from local machines to high-performance computing clusters, performing a comprehensive analysis without the requirement of extensive bioinformatics expertise.

Using DEN-IM, we successfully analysed two DENV datasets. The first comprised 25 shotgun metagenomic sequencing samples of variable serotype and genotype, including an in vitro spiked sample containing the four known serotypes. The second dataset consisted of 106 targeted metagenomic sequences of DENV 3 genotype III where DEN-IM allowed detection of the intra-genotype diversity.
The DEN-IM workflow, parameters and execution configuration files, and documentation are freely available at \url{https://github.com/B-UMMI/DEN-IM}.

\subsection{Keywords}
dengue virus, surveillance, metagenomics, reproducibility, workflow, containerization, scalability
 
\section{Author Notes}
All supporting data, code and protocols have been provided within the article or through supplementary data files.

Metagenomic sequencing data available under BioProject PRJNA474413. DEN-IM reports for the analysed datasets are available in Figshare under https://doi.org/10.6084/m9.figshare.11316599.v1. Phylogeny inference trees for the dengue virus typing database available in Figshare at https://doi.org/10.6084/m9.figshare.11316599.v1. The supplemental material is available in Figshare at https://doi.org/10.6084/m9.figshare.11316599.v1. DEN-IM’s source code and documentation available at https://github.com/B-UMMI/DEN-IM.

\section{Abbreviations}
 
\ac{CDS}; \ac{DENV}; \ac{HPC}; \ac{HTS}; NCR, non-coding region; QC, quality control; RT-PCT, reverse transcription polymerase chain reaction.

\section{Data Summary}

\begin{enumerate}
    \item The 106 DENV-3 targeted metagenomics paired-end short-read datasets are available under BioProject PRJNA394021. The 25 shotgun metagenomics dataset is available under BioProject PRJNA474413. The accession number for all the samples in the shotgun metagenomics dataset are available in the Supplementary material
    \item The accession numbers for the 41 samples, belonging to zika virus, chikungunya virus and yellow fever virus shotgun and targeted metagenomic datasets are available in the Supplementary material. 
    \item Code for the DEN-IM workflow is available at \url{https://github.com/B-UMMI/DEN-IM} and documentation, including step-by-step tutorials, is available at \url{https://github.com/B-UMMI/DEN-IM/wiki}.
\end{enumerate}

\section{Impact Statement}
The risk of exposure to DENV is increasing not only by travelling to endemic regions, but also due to the broader dissemination of the mosquito, making the burden of dengue very significant.

The decreasing costs and wider availability of HTS makes it an ideal technology to monitor DENV’s transmission. Metagenomics approaches decrease the time to obtain nearly complete DENV sequences without the need for time-consuming viral culture through the direct processing and sequencing of patient samples. A ready to use bioinformatics workflow, enabling the reproducible analysis of DENV, is therefore particularly relevant for the development of a straightforward HTS workflow.

DEN-IM was designed to perform a comprehensive analysis in order to generate either assemblies or consensus of full DENV CDSs and to identify their serotype and genotype. DEN-IM can also detect all four DENV genotypes present in a spiked sample, raising the possibility that DEN-IM can play a role in the identification of co-infection cases whose prevalence is increasingly appreciated in highly endemic areas. Although being ready-to-use, the DEN-IM workflow can be easily customised to the user’s needs.

DEN-IM enables reproducible and collaborative research, being accessible to a wide group of researchers regardless of their computational expertise and resources available.

\section{Introduction}

The Dengue virus (DENV), a single-stranded positive-sense RNA virus belonging to the Flavivirus genus, is one of the most prevalent arboviruses and is mainly concentrated in tropical and subtropical regions. Infection with DENV results in symptoms ranging from mild fever to haemorrhagic fever and shock syndrome \citep{organization_dengue_2009}. Transmission to humans occurs through the bite of Aedes mosquitoes, namely Aedes aegypti and Aedes albopictus \citep{diamond_molecular_2015}. In 2010, it was predicted that the burden of dengue disease reached 390 million cases/year worldwide \citep{bhatt_global_2013}. The high morbidity and mortality of dengue makes it the arbovirus with the highest clinical significance \citep{lourenco_challenges_2018}. DENV is a significant public health challenge in countries where the infection is endemic due to the high health and economic burden. Despite the emergence of novel therapies and ecological strategies to control the mosquito vector, there are still important knowledge gaps in the virus biology and its epidemiology \citep{diamond_molecular_2015}.

The viral genome of ~11,000 nucleotides, consists of a CDS of approximately 10.2 Kb that is translated into a single polyprotein encoding three structural proteins (capsid - C, premembrane - prM, envelope - E) and seven non-structural proteins (NS1, NS2A, NS2B, NS3, NS4A, NS4B and NS5). Additionally, the genome contains two Non-Coding Regions (NCRs) at their 5’ and 3’ ends \citep{leitmeyer_dengue_1999}.

DENV can be classified into four serotypes (1, 2, 3 and 4), differing from each other from 25\% to 40\% at the amino acid level. They are further classified into genotypes that vary by up to 3\% at the amino acid level \citep{diamond_molecular_2015}. The DENV-1 serotype comprises five genotypes (I-V), DENV-2 groups six (I-VI, also named American, Cosmopolitan, Asian-American, Asian II, Asian I and Sylvatic), DENV-3 four (I-III and V), and DENV-4 also four (I-IV).

The implementation of a surveillance system relying on HTS technologies allows the simultaneous identification and surveillance of DENV cases. Due to the high sensitivity of these technologies, previous studies showed that viral sequences can be directly obtained from patient sera using a shotgun metagenomics approach \citep{yozwiak_virus_2012}. Alternatively, HTS can be used in a targeted metagenomics approach in which a PCR step is used to pre-amplify viral sequences before sequencing. In recent years, HTS has been successfully used as a tool for identification of DENV directly from clinical samples \citep{yozwiak_virus_2012, lee_clinical_2017}. This also allows the rapid identification of the serotype and genotype important for disease management as the genotype may be associated with disease outcome \citep{fatima_serotype_2011}.

Several initiatives aim to facilitate the identification of the DENV serotype and genotype from HTS data. The Genome Detective project (https://www.genomedetective.com/) offers an online Dengue Typing Tool (https://www.genomedetective.com/app/typingtool/dengue/) \citep{fonseca_computational_2019} relying on BLAST and phylogenetic methods in order to identify the closest serotype and genotype, but it requires as input assembled genomes in FASTA format. The same project also offers the Genome Detective Typing Tool (https://www.genomedetective.com/app/typingtool/virus/) \citep{vilsker_genome_2019} identifying viruses present in a sample. Additionally, there are several tools available for viral read identification and assembly, such as VIP \citep{li_vip_2016}, virusTAP \citep{yamashita_virustap_2016} and drVM \citep{lin_drvm_2017}, but none performs genotyping of the identified reads.

We developed DEN-IM as a ready-to-use, one-stop, reproducible bioinformatic analysis workflow for the processing and phylogenetic analysis of DENV using paired-end raw HTS data. DEN-IM is implemented in Nextflow \citep{di_tommaso_nextflow_2017}, a workflow manager software that uses Docker (https://www.docker.com) containers with pre-installed software for all the workflow tools. The DEN-IM workflow, as well as parameters and documentation, are available at \url{https://github.com/B-UMMI/DEN-IM}.

\section{The DEN-IM Workflow}

DEN-IM is a user-friendly automated workflow enabling the analysis of shotgun or targeted metagenomics data for the identification, serotyping, genotyping, and phylogenetic analysis of DENV, as represented in Figure \ref{}, accepting as input raw paired-end sequencing data (FASTQ files) and informing the user with an interactive and comprehensive HTML report (Supplementary Figure \ref{}), as well as providing output files of the whole pipeline. 

It is implemented in Nextflow, a workflow management system that allows the effortless deployment and execution of complex distributed computational workflows in any UNIX-based system, from local machines to \ac{HPC} with a container engine installation, such as Docker (https://www.docker.com/), Shifter \citep{gerhardt_shifter_2017} or Singularity \citep{kurtzer_singularity_2017}. DEN-IM integrates Docker containerised images, compatible with other container engines, for all the tools necessary for its execution, ensuring reproducibility and the tracking of both software code and version, regardless of the operating system used. 

Users can customise the workflow execution either by using command line options or by modifying the simple plain-text configuration files. To make the execution of the workflow as simple as possible, a set of default parameters and directives is provided. An exhaustive description of each parameter is available as Supplementary material (see \ref{}).

The local installation of the DEN-IM workflow, including the docker containers with all the tools needed and the curated DENV database, requires 15 Gigabytes (Gb) of free disk space. The minimum requirements to execute the workflow are at least 5 Gb of memory and 4 CPUs. The disk space required for execution depends greatly on the size of the input data, but for the datasets used in this article, DEN-IM generates approximately 5 Gb of data per Gb input data.
DEN-IM workflow can be divided into the following components:

\subsubsection{Quality Control and Trimming}

The Quality Control (QC) and Trimming block starts with a process to verify the integrity of the input data. If the sequencing files are corrupted, the execution of the analysis of that sample is terminated. The sequences are then processed by FastQC (https://www.bioinformatics.babraham.ac.uk/projects/fastqc/, version 0.11.7) to determine the quality of the individual base pairs of the raw data files. The low-quality bases and adapter sequences are trimmed by Trimmomatic \citep{schmieder_quality_2011} (version 0.36). In addition, paired-end reads with a read length shorter than 55 nucleotides after trimming are removed from further analyses. Lastly, the low complexity sequences, containing over 50\% of poly-A, poly-N or poly-T nucleotides, are filtered out of the raw data using PrinSeq \citep{schmieder_quality_2011} (version 0.10.4).

\subsubsection{Retrieval of DENV sequences}

In the second step, DENV sequences are selected from the sample using Bowtie2 \citep{langmead_fast_2012} (version 2.2.9) and Samtools \citep{langmead_fast_2012} (version 1.4.1). As a reference we provide the DENV mapping database, a curated DENV database composed of 3830 complete DENV genomes. An in-depth description of this database is available as Supplementary material (see \ref{}). A permissive approach is followed by allowing for mates to be kept in the sample even when only one read maps to the database in order to keep as many DENV derived reads as possible. The output of this block is a set of processed reads of putative DENV origin.

\subsubsection{Assembly}

DEN-IM applies a two-assembler approach to generate assemblies of the DENV CDS. To obtain a high confidence assembly, the processed reads are first de novo assembled with SPAdes \citep{bankevich_spades_2012} (version 3.12.0). If the full CDS fails to be assembled into a single contig, the data is re-assembled with the MEGAHIT assembler \citep{li_megahit_2015} (version 1.1.3), a more permissive assembler developed to retrieve longer sequences from metagenomics data. The resulting assemblies are corrected with Pilon \citep{walker_pilon_2014} (version 1.22) after mapping the processed reads to the assemblies with Bowtie2.

If more than one complete CDS is present in a sample, each of the sequences will follow the rest of the DEN-IM workflow independently. If no full CDS is assembled neither with SPAdes nor with MEGAHIT, the processed reads are passed on to the next module for consensus generation by mapping, effectively constituting DEN-IM’s two-pronged approach using both assemblers and mapping.

\subsubsection{Typing}

For each DENV complete CDS, the serotype and genotype is determined with the Seq\_Typing tool (\url{https://github.com/B-UMMI/seq\_typing}, version 2.0) \citep{machado_epidemiological_2017} using BLAST \citep{altschul_gapped_1997} and the custom Typing database of DENV containing 161 complete sequences (see \ref{}). The tool determines which reference sequence is more closely related to the query based on the identity and length of the sequence covered, returning the serotype and genotype of the reference sequence

If a complete CDS fails to be obtained through the assembly process, the processed reads are mapped against the same DENV typing database, with Bowtie2, using the Seq\_Typing tool, with similar criteria for coverage and identity to those used with the BLAST approach. If a type is determined, the consensus sequence obtained follows through to the next step in the workflow. Otherwise, the sample is classified as Non-Typable and its process terminated.

\subsubsection{Typing}

All DENV complete CDSs and consensus sequences analysed in a workflow execution are aligned with MAFFT \citep{nakamura_parallelization_2018} (version 7.402). By default, or if the number of samples analysed is less than 4, four representative sequences for each DENV serotype (1 to 4) from NCBI are also included in the alignment. The NCBI references included are NC\_001477.1 (DENV-1), NC\_001474.2 (DENV-2), NC\_001475.2 (DENV-3) and NC\_002640.1 (DENV-4). The closest reference sequence to each analysed sample in the DENV typing database to each analysed sample can also be retrieved and included in the alignment. With the resulting alignment, a Maximum Likelihood tree is constructed with RaXML \citep{stamatakis_raxml_2014} (version 8.2.11). 

\subsubsection{Output and Report}

The output files of all tools in DEN-IM’s workflow are stored in the ’results’ folder in the directory of DEN-IM’s execution, as well as the execution log file DEN-IM and for each component. 

The HTML report (Supplementary Figure \ref{}), stored in the ’pipeline\_results’ directory contains all results divided into four sections: report overview, tables, charts and phylogenetic tree. The report overview and all tables allow for selection, filtering and highlighting of particular samples in the analysis. All tables have information on if a sample failed or passed the quality control metrics highlighted by green, yellow or red signs for pass, warning and fail messages, respectively. 

The \textit{in silico} typing table contains the results of the serotype and genotype of each CDS analysed, as well as identity, coverage and GenBank ID of the closest reference in the DENV typing database. The quality control table shows information regarding the number of raw base pairs and number of reads in the raw input files and the percentage of trimmed reads. The mapping table includes the results for the mapping of the trimmed reads to the DENV mapping database, including the overall alignment rate, and an estimation of the sequence depth including only the DENV reads. For the assembly statistics table, the number of CDSs in each sample, the number of contigs and the number of assembled base pairs generated by either SPAdes or MEGAHIT assemblers is included. The number of contigs and assembled base pairs after correction with Pilon is also presented in the table. The assembled contig size distribution scatter plot is available in the chart section, showing the contig size distribution for the Pilon corrected assembled CDSs.

Lastly, a phylogenetic tree is included, rooted at midpoint for visualisation purposes, and with each tip coloured according to the genotyping results. If the option to retrieve the closest typing reference is selected, these sequences are also included in the tree with respective typing metadata. The tree can be displayed in several conformations provided by Phylocanvas JavaScript library (\url{http://phylocanvas.net}, version 2.8.1) and it is possible to zoom in or collapse selected branches. The support bootstrap values of the branches can be displayed, and the tree can be exported as a Newick tree file or as a PNG image.

\section{Software comparison}

DEN-IM offers a core assembly functionality, leveraging a de novo and consensus assembly approach, to obtain a full CDS sequence to perform geno- and serotyping, followed by phylogenetic positioning of the samples analysed. This results in a phylogenetic tree showing the genotyping results, presented in an HTML file.

There are several alternative tools, both command line and online based, capable of identifying DENV reads and performing assembly (Table \ref{}). VIP and drVM are both stand-alone pipelines, like DEN-IM, and several components overlap with DEN-IM’s but the retrieval of viral sequences is not targeted for DENV, and no serotyping and genotyping is performed. VIP performs a phylogenetic analysis against the reference database. VirusTAP is a web server for the identification of viral reads using the ViPR and IRD databases, or alternatively with the RefSeq Virus database. GenomeDetective is also a web service that provides two tools, one for the assembly of viral sequences from raw data (Virus tool) and another for serotyping and genotyping of DENV fasta sequences (Dengue Typing tool). Both tools need to be run consecutively, with the Virus Tool providing a link to redirect to the Dengue Typing tool when a DENV sequence is identified.

Of all the tools listed in Table \ref{}, only Genome Detective offers a tool to determine the DENV sero- and genotype from a fasta sequence, but the need to run their virus identification tool prior to obtain a sequence from the raw sequencing data increases the time to obtain a typing result, especially when a large number of sequences needs to be analysed. Moreover, these tools are not open source, so we are unable to compare the methodology used with our own. Additionally, there might be privacy issues in submitting data to external services, like VirusTAP and GenomeDetective, especially when handling metagenomics data that contain human sequences subjected to strict privacy laws in most countries. Therefore, a stand-alone tool is preferable for these analyses since these can be run in secure local environments. DEN-IM’s main advantage when compared to web-based platforms is the ability to analyse batches of samples in a scalable manner, obtaining a report summarizing all the samples analysed and a phylogeny analysis of all DENV CDSs recovered.