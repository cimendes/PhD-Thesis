\mbox{}\\
\vspace{8cm}

The work I hereby present aims to evaluate the use of bioinformatics methods for the analysis of metagenomic data to allow the rapid identification, virulence analysis and antimicrobial susceptibility prediction of pathogens with clinical relevance. \ac{SMg} still presents as a promising methodology to obtain very fast results for the identification of pathogens and their virulence and resistance properties directly from samples, without the need for culture. 
To be accredited and used in clinical settings, this approach must be standardised and the statistical metrics used to analyse and report the data must be validated.

The impact and applicability of \ac{SMg} in clinical microbiology, including both a diagnosis and surveillance and infection prevention, has been assessed, with the unique challenges of both highlighted. For diagnostics, the biggest drawback of this methodology is the extremely cost-ineffective negative results, allied with the very high sensitivity that translates into several false-positive results. It has been used as a last-resort diagnostic technique \citep{he_case_2022, vijayvargiya_application_2019, sanabria_shotgun-metagenomics_2020, hirakata_application_2021}, with great success when classical approaches fail to detect the causative agent of a disease. Chapter \ref{ch:paper1} aimed to evaluate \ac{SMg} approaches in a collection of varied samples, with particular interest in the bioinformatic pipeline applied, and how it affected the results obtained in a potential diagnosis setting. Most pathogens identified by culture were also identified through metagenomics, but substantial differences were noted between the taxonomic classification tools. In surveillance and infection prevention, \ac{SMg} has also been successfully applied \citep{loman_culture-independent_2013, huang_metagenomics_2017, li_microbiome_2021}, with programs existing relying on this methodology for the monitoring of emerging or of global interest pathogens \citep{ko_metagenomics-enabled_2022}. Chapter \ref{ch:paper2}, the novel detection of an \textit{mcr}-5 gene, named \textit{mcr}-5.4, is reported, from a concentrated water sample. 

The accurate identification of pathogens of interest from \ac{SMg} remains one of the biggest challenges when analysing this type of data. A hybrid approach of read mapping and \textit{de novo} assembly methods sometimes proved the only way to successfully recover sequences of interest with enough quality for genome reconstruction. In Chapter \ref{ch:paper3}, both approaches were employed to provide, in a single methodological step, identification and characterization of a whole viral genome at the nucleotide level. Furthermore, Chapter \ref{ch:paper4} highlighted that no assembler stood out as an undisputed all-purpose choice for short-read metagenomic prokaryote genome assembly, hence efforts are still needed to further improve metagenomic assembly performance. 

A strong focus on  the standardisation and reproducibility of the results obtained, with the employment of new technologies to do so, such as container software and workflow managers, is of the uttermost necessity for the \ac{SMg} data analysis solutions. Transparency, scalability, and ease of installation are key-stones, regardless of the tools chosen. The solutions adopted throughout this work, such the use of docker, nextflow and conda, allied to clear and easy to follow documentation, aim to lower the barrier of entry when performing detailed analysis that are complex and computationally expensive in their nature. But most importantly, the production of intuitive, responsive and easy-to-follow reports, allowing the summary of key results, as well as the detailed exploration of the resulting data, by stakeholders, be it bioinformatic personnel or experts in the given area of expertise, represent the single most important contribution to lowering the barrier between who produces the data and who has the capacity to make informed decisions based on that data.   

As public health laboratories expand their genomic sequencing and bioinformatics capacity for the surveillance of different pathogens, labs must carry out robust validation, training, and optimisation of wet- and dry-lab procedures. Despite this richness in genomic information, the same is not observed for the contextual information that accompanies it. This contextual information, composed of metadata  Standardisation of harmonisable, publicly available community standards is ultimately what will allow the transmission of information through various stakeholders, containing the minimum information required for it to be understandable and actionable across domains. 

\section{Future perspectives}

Despite the results presented in this work, the future of \ac{SMg} still looks promising. Regardless, for it to pass from the last resort solution when traditional methods fail to a standard in clinical microbiology, some issues still need to be addressed. 

One of the biggest hinders to the application of \ac{SMg} in the clinic is the significant cost of negative results, allied with a very high sensitivity but not as high specificity, being very prone to contamination leading to false positive results. Therefore, part of the work to improve this technology passes through the improvement in the bench. Host depletion steps still have a long way to produce satisfactory results for diagnosis purposes, without altering the proportions of organisms in a community. Although traditional methods require a very significant amount of man-hours, sequencing requires a different, more specialised set of skills to obtain quality results. Additionally, the time gained in the results attainment is offset by the increased cost. 

Regarding bioinformatic analysis of \ac{SMg} data, the lack of golden standards presents the biggest hindrance. But as new methods keep being developed, fuelled by the extreme and exciting interest of the community for this methodology, and more importantly, as current methods continue to be validated, and their results openly shared, in a faindable and FAIRer way, we slowly but surely walk towards the standardisation of methodologies. 

I'm as excited with the potential of \ac{SMg} for clinical microbioly, both in the diagnostics and public health and surveillance branches, as I was when I first started this project. The potentials are staggering, albeit currently it's still very much an experimental and/or last resort approach. As methodologies continue to improve, sequencing will become ubiquitous in every diagnostics and public health laboratory. The transition between academic applications to being a viable solution in hospitals and health reference laboratories might be slow, or at least slower than what was once expected, but it's an inevitability that we should not only be prepared for but also excited about. 