By analysing metagenomic data, the present thesis aims to identify pathogens with clinical relevance, analyse their virulence, and predict their antimicrobial susceptibility. Metagenomic methods must be evaluated in a clinical setting as an alternative to current golden standards. Given the dependence of these methodologies on bioinformatics post-processing of the raw data obtained, the significant applications and pitfalls of metagenomics are yet to be identified.

Microbial pathogens are responsible for more than 400 million years of life lost annually across the globe, a higher burden than either cancer or cardiovascular disease. In addition to the emergence of virulent pathogens, the rise of antimicrobial resistance poses a major threat to human health worldwide. Clinical microbiology is a discipline focused on rapidly characterising pathogen samples to direct the management of individual infected patients (diagnostic microbiology) and monitor the epidemiology of the infectious disease (public health microbiology), including the detection of outbreaks and infection prevention. For the purpose of this dissertation work, we will focus on bacterial and viral infections.

Since the publication of the first complete microbial genome, a quarter of a century ago genomics has transformed the field of microbiology, and in particular its clinical application. The development and commercialisation of high-throughput, massively parallel sequencing has democratised sequencing by offering individual laboratories, either in research or in health, access to the technology. Three main revolutions can be considered in
genomic sequencing: the first generation, also known as automated Sanger sequencing, the second generation, also known as next-generation sequencing, and the third generation of sequencing, also known as long-read sequencing or single-molecule sequencing. Whole genome sequencing has been used in the routine laboratory workflow when typing pathogens by a method having the highest possible discriminatory power in comparison with the golden-standard molecular methods, such as polymerase chain reaction. Most notably, whole genome sequencing has become a common tool in infection surveillance and prevention, allowing the identification and tracking of pathogens, establishing transmission routes and outbreak control. Despite this, the implementation of whole genome sequencing in routine diagnostics requires several adaptations in the laboratory workflow, from the ‘wet’ laboratory part (extraction, library preparation, sequencing), to the ‘dry’ bioinformatics part where genomic data is analysed and its results interpreted by specialised personnel. 

It is true that DNA sequencing methods are increasingly being adopted in clinical microbiology, but it does require a priori knowledge of what a clinical sample or patient will contain. One of the possibilities for overcoming this limitation is through the implementation of metagenomics, delivering culture-independent approaches to microbial ecology, surveillance and diagnosis. While most molecular assays target only a limited number of pathogens, metagenomic approaches characterise all DNA or RNA present in a sample, enabling analysis of the entire microbiome as well as the human host genome or transcriptome in patient samples. Whether or not it can entirely replace routine microbiology depends on several conditions and future developments, both technological and computational. The bioinformatics analysis, required due to the amount of data produced by genomic sequencing technologies, represents one of the major cornerstones for the applicability of this methodology in diagnosis and surveillance. 

In this work, shotgun metagenomics was successfully applied to nine body fluid samples and one tissue sample from patients at the University Medical Center Groningen with varying degrees of contamination. Shotgun metagenomics was compared to standard culture-based microbiological methods in order to evaluate and compare the accuracy and reliability of the bioinformatics analyses. Furthermore, this methodology was applied to eight concentrated water samples also collected from the University Medical Center Groningen. In one of the samples, the novel detection of an \textit{mcr-5} gene, named \textit{mcr-5.4}, is reported.

With the lessons learned from processing both the clinical and environmental samples collected, we have developed DEN-IM, a one-stop, user-friendly, containerised and reproducible workflow for the analysis of Dengue virus short-read sequencing data from both amplicon and shotgun metagenomics approaches. Genomic sequencing is the most informative approach to monitor viral dissemination and genetic diversity by providing, in a single methodological step, identification and characterization of the whole viral genome at the nucleotide level. DEN-IM was designed to perform a comprehensive analysis in order to generate either assemblies or consensus of full DENV coding sequences and to identify their serotype and genotype. DEN-IM can play a role in the identification of co-infection cases whose prevalence is increasingly perceived in highly endemic areas.

Shotgun metagenomics can offer comprehensive microbial detection and characterisation of
complex clinical samples. The de novo assembly of raw sequence data is key in metagenomic analysis, yielding longer sequences that offer contextual information and afford a more complete picture of the microbial community. The assembly process is the bedrock and may constitute a major bottleneck in obtaining trustworthy, reproducible results. Hence, LMAS, an automated workflow, was developed as a flexible platform to allow users to evaluate traditional and metagenomic dedicated prokaryotic de novo assembly software performance given known standard communities. Similarly to DEN-IM, LMAS implementation ensures the transparency and reproducibility of the results obtained, presented in an interactive HTML report where global and reference-specific performance metrics can be explored. 

Despite the relative standardisation of the process to acquire genomic data, either through whole genome sequencing or metagenomics, there’s a myriad of different tools available to perform in silico antimicrobial resistance detection, with some using their own reference database and each one generating a unique, non-standardised report of the genes or variants that can possibly confer resistance in a given sample. This is a huge barrier to the comparison of results and the modularity of tools within bioinformatic workflows. Given this predicament, a standardised output specification for the reporting of genes or variants potentially conferring antimicrobial resistance is presented, packaged into hAMRonization, a command-line utility that is able to aggregate results from a wide variety of AMR detection tools, both species-agnostic and species-specific, providing a unified report tabular form, JSON or through an interactive HTML file. 

Despite the richness of genomic information available, the same is not observed for the contextual information that accompanies it. Following the same approach developed for antimicrobial resistance detection, a standardised output specification was conceived, this time applied to SARS-CoV-2 contextual data based on harmonisable, publicly available community standards. This is implemented through a collection template, as well as a variety of protocols and tools to support both the harmonisation and submission of sequence data and contextual information to public biorepositories. 

Throughout this work, it became obvious that computational algorithms have become an essential component of microbiome research, with great efforts by the scientific community to raise standards on the development and distribution of code. Despite these efforts, sustainability and reproducibility are major issues since continued validation through software testing is still not a widely adopted practice. In an effort to uphold good software engineering practices, we report seven recommendations that help researchers implement software testing in microbial bioinformatics. We propose collaborative software testing as an opportunity to continuously engage software users, developers, and students to unify scientific work across domains. As automated software testing remains underused in scientific software, our set of recommendations not only ensures appropriate effort can be invested into producing high-quality and robust software but also increases engagement in its sustainability. 

The impact and applicability of shotgun metagenomic in clinical microbiology, including both diagnosis and surveillance and infection prevention, has been assessed, with the unique challenges of both highlighted. A strong focus on the standardisation and reproducibility of the results obtained, with the employment of new technologies to do so is of the uttermost necessity for the shotgun metagenomic data analysis solutions. Transparency, scalability, and ease of installation are keystones, regardless of the tools chosen. The solutions adopted throughout this work, allied to clear and easy-to-follow documentation, aim to lower the barrier of entry when performing detailed analyses that are complex and computationally expensive in their nature. But most importantly, the production of intuitive, responsive and easy-to-follow reports, allows the summary of key results, as well as the detailed exploration of the resulting data, by stakeholders, be it bioinformatic personnel or experts in the given area of expertise, represent the single most important contribution to lowering the barrier between who produces the data and who has the capacity to make informed decisions based on that data. 