\mbox{}\\
\vspace{8cm}

This chapter is a reproduction of the following publication:

N. Couto, L. Schuele, E.C. Raangs, M. P. Machado, C. I. Mendes, T. F. Jesus, M. Chlebowicz,  S. Rosema, M. Ramirez, J. A. Carriço, I. B. Autenrieth, A. W. Friedrich, S. Peter and J. W. Rossen. Critical steps in clinical shotgun metagenomics for the concomitant detection and typing of microbial pathogens. Sci Rep 8, 13767 (2018). \url{https://doi.org/10.1038/s41598-018-31873-w}

The supplementary information referred throughout the text can be consulted in
this chapter before the section of references. 

My contribution to this publication included....

\cleardoublepage 

\begin{center}
\large
\textbf{Critical steps in clinical shotgun metagenomics for the concomitant detection and typing of microbial pathogens}
\end{center}

Natacha Couto$^1$, 
Leonard Schuele$^1,2$,
Erwin C. Raangs$^1$,
Miguel P. Machado$^3$, 
Catarina I. Mendes$^1,3$,
Tiago F. Jesus$^3$, 
Monika Chlebowicz$^1$, 
Sigrid Rosema$^1$, 
Mário Ramirez$^3$, 
João A. Carriço$^3$, 
Ingo B. Autenrieth$^2$, 
Alex W. Friedrich$^1$, 
Silke Peter$^2$, 
John W. Rossen$^1$

$^1$ University of Groningen, University Medical Center Groningen, Department of Medical Microbiology, Groningen, The Netherlands;

$^2$ Institute of Medical Microbiology and Hygiene, University of Tübingen, Germany; 

$^3$ Instituto de Microbiologia, Instituto de Medicina Molecular, Faculdade de Medicina, Universidade de Lisboa, Portugal.


\section{Abstract} \label{sec:ch2_abstract}

High throughput sequencing has been proposed as a one-stop solution for diagnostics and molecular typing directly from patient samples, allowing timely and appropriate implementation of measures for treatment, infection prevention and control. 
However, it is unclear how the variety of available methods impacts the end results. 
We applied shotgun metagenomics on diverse types of patient samples using three different methods to deplete human DNA prior to DNA extraction.
Libraries were prepared and sequenced with Illumina chemistry. 
Data was analysed using methods likely to be available in clinical microbiology laboratories using genomics. 
The results of microbial identification were compared to standard culture-based microbiological methods. 
On average, 75\% of the reads were corresponded to human DNA, being a major determinant in the analysis outcome. 
None of the kits was clearly superior suggesting that the initial ratio between host and microbial DNA or other sample characteristics were the major determinants of the proportion of microbial reads. 
Most pathogens identified by culture were also identified through metagenomics, but substantial differences were noted between the taxonomic classification tools. 
In two cases the high number of human reads resulted in insufficient sequencing depth of bacterial DNA for identification. 
In three samples, we could infer the probable multilocus sequence type of the most abundant species. 
The tools and databases used for taxonomic classification and antimicrobial resistance identification had a key impact on the results, recommending that efforts need to be aimed at standardisation of the analysis methods if metagenomics is to be used routinely in clinical microbiology.

\section{Introduction} \label{sec:ch2_introduction}

Classical microbial culture is still considered the gold standard in medical microbiology. 
Several molecular detection techniques have been implemented but these are generally geared towards specific pathogens (e.g. specific RT-PCR or microarrays). 
Even when unbiased molecular approaches are used, such as 16S/18S rRNA gene sequencing, these do not provide all the information that can be obtained by culturing, e.g., antimicrobial susceptibility and molecular typing information. 
However, microbial culture is laborious and time-consuming and new methods are needed to replace it. 
Ideally, a single method should provide rapid identification and characterisation of clinically relevant pathogens directly from a sample in order to guide therapy, predict potential treatment failures and to reveal possible transmission events.

Shotgun metagenomics (SMg) is a culture-independent technique that provides valuable information not only at the identification level, but also at the level of molecular characterisation. 
Studies have shown that it has added value in terms of detection sensitivity and personalised treatment in clinical microbiology, when identifying bacteria \citep{hasman_rapid_2014, willmann_antibiotic_2015} or viruses \citep{graf_unbiased_2016}. 
Indeed Gyarmati et al., 2016 \citep{gyarmati_metagenomic_2016}, used a sequence-based metagenomics approach directly from blood to detect non-culturable, difficult-to-culture and non-bacterial pathogens. 
The authors were able, through SMg, to detect viral and fungal pathogens together with bacteria, which had not been detected through classical microbiology. 
Additionally, SMg can be used for infection prevention, having the potential to identify transmission events directly from clinical samples \citep{olson_metagenomic_2017}. 
For example, SMg was proven valuable for the identification of inter-host nucleotide variations occurring after direct transmission of noroviruses causing gastroenteritis \citep{olson_metagenomic_2017}. 
Hasman and colleagues (2014) \citep{hasman_rapid_2014} were able to identify urinary pathogens directly from urine, as well as antimicrobial resistant genes compatible with the resistant phenotype determined through antimicrobial susceptibility testing. They also identified almost perfect phylogenetic matches between whole-genome sequence (WGS) data obtained by metagenomics and WGS of pure isolates. 

Despite the promise of SMg of becoming a one-stop solution in clinical microbiology, SMg still has several challenges to overcome. 
One of the greatest challenges is the choice of the extraction and sequencing protocols, as well of the type of controls \citep{schlaberg_validation_2017}. 
The extraction protocol should efficiently and specifically isolate microbial DNA/RNA, while removing the host DNA/RNA \citep{street_molecular_2017}. 
However, the variety of clinical samples used in the diagnosis of distinct types of infection (e.g. tissues versus fluids), poses a serious challenge for standardisation, an essential step if these methods are to be used by routine diagnostic laboratories. 
The sequencing protocol is also dependent on the pathogens of interest (e.g. bacteria versus viruses), sequencing strategy (DNA and/or RNA), required turnaround time, sequencing depth and error tolerance \citep{schlaberg_validation_2017}. The use of defined controls is necessary for validation of each experiment and these should be adapted for every type of infection and sample type and should consist of a combination of known positive specimens, pathogen-negative patient specimens and pathogen-negative patient specimens spiked with live microorganisms or pure DNA \citep{schlaberg_validation_2017}.

Another potential challenge are the metagenomics analysis tools. 
Recent studies have evaluated the different SMg sequence classification methods \citep{peabody_evaluation_2015}. 
These use different methodologies for classification: sequence similarity-based methods, sequence composition-based methods and hybrid methods \citep{peabody_evaluation_2015}. 
They differ not only in the algorithms for detecting the microorganisms present, but also in the databases used. 
This high variability leads to different results, not only at the microorganism classification level but also when evaluating the relative abundance of these pathogens \citep{peabody_evaluation_2015}. 
A recent study evaluated the accuracy of 38 bioinformatics methods using both \textit{in silico} and \textit{in vitro} generated mock bacterial communities.
Dozens to hundreds of species were falsely predicted by the most popular software, and no software clearly outperformed the others \citep{peabody_evaluation_2015}. 
In the absence of studies comparing the outputs of different analysis methods in clinical samples, users may decide which methods to use based on personal experience with a given tool, availability of the tool in the laboratory or its ease of use.
This poses a great challenge when providing reproducible results and creates uncertainty regarding the reliability of the information derived. This is a major barrier to the implementation of SMg approaches in routine clinical microbiology laboratories.

In this study, the aim was to identify the critical steps when using SMg for the identification and characterization of microbial pathogens directly from clinical specimens using methods that are likely to be available in clinical microbiology laboratories wanting to implement genomics for pathogen identification or molecular epidemiology studies. 
For this purpose, we used three human-DNA depletion kits and evaluated a diverse set of bioinformatics tools (commercial and non-commercial) in order to investigate how well they performed and what would the differences be in terms of taxonomic classification, antimicrobial resistance gene detection and typing directly from patient samples, bypassing culture. 

\section{Methods} \label{sec:methods}

\subsection{Sample collection} \label{ssec:sample_collection}

Nine body fluid samples and one tissue sample entering the Medical Microbiology laboratory were selected for metagenomics sequencing. 
These included one sample from peritoneal fluid, five from pus (3 abscesses and 2 empyema), two from synovial fluid of knees with prosthesis, one from sputum and one from a bone biopsy (Table \ref{tab:table_1}). 
All samples were stored at 4ºC for a variable period (2-10 days). 
The samples used for the present analyses were collected during routine diagnostics and infection prevention and control investigations. 
All procedures were carried out according to guidelines and regulations of University Medical Centre Groningen (UMCG) concerning the use of patient materials for the validation of clinical methods, which are in compliance with the guidelines of the Federation of Dutch Medical Scientific Societies (FDMSS).
Every patient entering the UMCG is informed that samples taken may be used for research and publication purposes, unless they indicate that they do not agree to it. 
This procedure has been approved by the Medical Ethical Committee of the UMCG. 
Informed consent was obtained from all individuals or their guardians prior to study participation. 
All samples were used after performing and completing a conventional microbiological diagnosis and were coded to protect patients’ confidentiality. 
All experiments were performed in accordance with the guidelines of the Declaration of Helsinki and the institutional regulations. 

\begin{table}[]
\caption{Characteristics of the samples and mapping of trimmed reads against a human genome hg19 (\%) using CLC Genomics Workbench v10.0.1.}
\label{tab:ch2_table_1}
\resizebox{\textwidth}{!}{%
\begin{tabular}{@{}llllll@{}}
\toprule
\textbf{Sample} & Sample type & DNA extraction method & Total number of reads & Mapped reads against hg19 & \textbf{Unmapped reads} \\ \midrule
Sample 1         & Peritoneal fluid & Ultra-Deep Microbiome Prep (Molzym) & 5892978 & 5,249,063 (89.2\%) & 632,951 (10.8\%)   \\
Sample 2         & Pus (abscess)    & Ultra-Deep Microbiome Prep (Molzym) & 9603346 & 7,828.746 (81.6\%) & 1,770,558 (18.4\%) \\
Sample 3         & Synovial fluid   & Ultra-Deep Microbiome Prep (Molzym) & 8615810 & 8,254,594 (95.9\%) & 355,200 (4.1\%)    \\
Sample 4         & Synovial fluid   & Ultra-Deep Microbiome Prep (Molzym) & 6078166 & 6,015,945 (99.0\%) & 61,099 (1.0\%)     \\
Sample 5         & Pus (abscess)    & Ultra-Deep Microbiome Prep (Molzym) & 8368930 & 309,588 (3.7\%)    & 8,052,272 (96.3\%) \\
Sample 6         & Pus (empyema)    & QIAamp DNA Microbiome Kit (Qiagen)  & 2912802 & 2,877,066 (98.8\%) & 34,506 (1.1\%)     \\
Sample 7         & Pus (empyema)    & QIAamp DNA Microbiome Kit (Qiagen)  & 1486700 & 922,932 (62.2\%)   & 561,772 (37.8\%)   \\
Sample 8         & Bone biopsy      & Micro-DXTM (Molzym)                 & 6534866 & 229,149 (3.5\%)    & 6,303,803 (96.5\%) \\
Sample 9         & Pus (abscess)    & Micro-DXTM (Molzym)                 & 6173132 & 6,081,612 (98.5\%) & 89,922 (1.5\%)     \\
Sample 10        & Sputum           & Micro-DXTM (Molzym)                 & 7596836 & 7,337,832 (96.7\%) & 235,520 (3.3\%)    \\
Negative control & Water            & QIAamp DNA Microbiome Kit (Qiagen)  & 1730738 & 1,706,861 (98.9\%) & 19,805 (1.2\%)     \\ \bottomrule
\end{tabular}%
}
\end{table}

\subsection{Classic culturing and susceptibility testing} \label{ssec:sample_culturing}

The samples were cultured following methods routinely used in our institution. 
Briefly, samples were streaked onto five plates (Mediaproducts BV, Groningen, The Netherlands) - blood agar (aerobic), chocolate agar (aerobic), McConkey agar (aerobic), Brucella agar (anaerobic) and Sabouraud Dextrose +AV (aerobic) - and incubated overnight under aerobic and anaerobic atmosphere at 37ºC. 
The two pus samples were also plated onto Phenylethyl alcohol sheep blood agar (PEA), Kanamycin vancomycin laked blood (KVLB) agar and Bacteroides bile esculin (BBE) agar and incubated under anaerobic conditions overnight. 
The isolates recovered were subjected to susceptibility testing by Vitek 2 using either the AST-P559 (Gram-positive bacteria) or the AST-N344 (Gram-negative bacteria) card (bioMérieux, Marcy-l'Étoile, France) and identified by MALDI-TOF MS (Bruker Daltonik, Gmbh, Germany) using standard protocols. 

\subsection{DNA extraction, library preparation and sequencing} \label{ssec:sample_sequencing}

The DNA for metagenomic sequencing was isolated using the Ultra-Deep Microbiome Prep (Molzym Life Science, Bremen, Germany), Micro-Dx\texttrademark kit (Molzym Life Science) or QIAamp DNA Microbiome Kit (Qiagen, Hilden, Germany) directly from the clinical samples and a negative control consisting of a mock sample of DNA and RNA free water (Table \ref{tab:table_1}). 
These kits include human DNA depletion steps. 
The QIAamp DNA Microbiome Kit was used according to the manufacturer’s protocol with an additional 5 min air-dry step before elution. 
For microbial lysis, a Precellys 24 homogeniser (Bertin, Montigny-le-Bretonneux, France) set to 3 times 30 seconds at 5000 rpm separated by 30 seconds was used. 
After extraction, DNA was quantified with the Qubit 2.0 (Life Technologies, ThermoFisher Scientific, Waltham, Massachusetts, EUA) and NanoDrop 2000 (ThermoFisher Scientific). 
The DNA quality was assessed using the Genomic DNA ScreenTape and Agilent 2200 TapeStation System (Agilent Technologies, California, United States of America). 
Isolated DNA was purified using Agencourt AMPure XP beads (Beckman Coulter, California, United States of America) according to the manufacturer’s instructions, to eliminate small DNA fragments and chemical contaminants (e.g. benzonase). 
The DNA was then diluted to 0.2 ng/$\mu$l and 1 ng was used for the library preparation, using the Nextera XT Library Preparation kit (Illumina, California, United States of America), according to the manufacturer’s protocol. 
Cluster generation and sequencing were performed with the MiSeq Reagent Kit v2 500-cycles Paired-End in a MiSeq instrument (Illumina). 
Samples were sequenced in batches of 5 samples on a single flow cell.

For the DNA extraction of bacterial isolates (when an isolate was recovered from culture), we used the UltraClean Microbial DNA Isolation Kit (Mo Bio), with some modifications. 
We started with solid cultures and resuspended a 10 $\mu$l-loopfull of culture directly into the tube with the microbeads and microbead solution. 
The library preparation, cluster generation and sequencing was performed as described above. 
Strains were sequenced in batches of 12 to 16 on a single flow cell.

\subsection{Bioinformatics analyses} \label{ssec:sample_bioinformatics}

In order to evaluate and compare the accuracy and reliability of the bioinformatics analyses in providing the closest results to culture and WGS of any cultured isolates, three different pipelines (two commercially and one freely available) were used (Figure 1).
Different tools to perform raw read quality control, filtering and trimming were used and reads were mapped against the human genome (hg19) before performing taxonomic classification. 
Reads mapping to hg19 where removed from the analysis to increase the efficiency of the bioinformatics tools. 
Typing (MLST), phylogenetic analysis, plasmid analysis, detection of antimicrobial resistance and virulence genes was performed. 
To determine the appropriateness of SMg as predictor of the WGS (chromosome and plasmids), SMg results obtained were compared with the results of WGS of any bacterial isolates obtained from culturing the sample.

All the parameters used in each approach are available in Supplementary Table 1.

\subsubsection{Unix-based approach}

For the metagenomics data, read quality control and cleaning was performed using FastQC v0.11.5 and Trimmomatic v0.36, respectively, through the INNUca v2.6 pipeline\footnote{\url{https://github.com/B-UMMI/INNUca/}}, excluding assembly and polishing. 
Using a reference mapping approach against the human genome (UCSC hg19), human reads were discarded using Bowtie 2 v2.3.2 \citep{langmead_fast_2012} and SAMtools v1.3.1 \citep{li_sequence_2009}. 
Those paired reads that did not map against the human genome were used in subsequent analyses. 
The bacterial species were identified through Kraken v0.10.5-beta \citep{wood_kraken_2014} using the miniKraken database (pre-built 4 GB database constructed from complete bacterial, archaeal and viral genomes in RefSeq, as of Dec. 8, 2014), MIDAS \citep{nayfach_integrated_2016} using the midas\_db\_v1.2 database (>30,000 bacterial reference genomes, as of May 9, 2018) and MetaPhlAn2 v2.0 \citep{segata_metagenomic_2012} using the database provided by the tool ($\sim$13,500 bacterial and archaeal, $\sim$3,500 viral, and $\sim$110 eukaryotic reference genomes, as of May 9, 2018). 
The sequence type (ST) was obtained through metaMLST v1.1 \citep{zolfo_metamlst_2017} based on the metamlstDB\_2017. 
Antimicrobial resistance genes were detected using ReMatCh v3.2\footnote{\url{https://github.com/B-UMMI/ReMatCh/}}, a read mapping tool that uses Bowtie 2 v2.3.2 \citep{langmead_fast_2012} and the following rules for gene presence/absence: genes were considered present when $\geq$ 80\% of the reference sequence was covered and the sample sequence was $\geq$ 70\% identical to the one used as reference. 
For that, ResFinder database (2231 genes, downloaded on 29-06-2017) was used as reference and, due to the low coverage of microbial metagenomics samples, a minimal coverage depth of 1 read was set to consider a reference sequence position as covered (and therefore present in the sample data), as well as to perform base call (used for sequence identity determination).
Finally, the assembly was accomplished through SPAdes v3.10.1 \citep{bankevich_spades_2012}.

Plasmid detection was achieved by running the script PlasmidCoverage\footnote{\url{https://github.com/tiagofilipe12/PlasmidCoverage}}, using the plasmid sequences downloaded from NCBI RefSeq (\url{ftp://ftp.ncbi.nlm.nih.gov/genomes/refseq/plasmid/}, as of May 11, 2017). 
The script uses Bowtie 2 v2.2.9 \citep{langmead_fast_2012}, to map the pre-processed input reads against the plasmid database (Bowtie2 index for all plasmid sequences). 
For Bowtie 2 we used the ‘-k’ option, allowing each read to map to as many plasmid sequences as present in the NCBI RefSeq plasmid database (since plasmid sequences are modular) \citep{smillie_mobility_2010, barcia_identification_2011}. 
Then, this pipeline used SAMtools v1.3.1 \citep{li_sequence_2009} to estimate the coverage for each position, and reported the length of plasmid sequence covered (in percentage) and average depth (mean number of reads mapped against a given position in each plasmid). 
Plasmids with less than 80\% of its length covered were excluded from the final results in line with what has described elsewhere \citep{jitwasinkul_plasmid_2016}. The pATLAS tool\footnote{\url{http://www.patlas.site/}} was used to visualise which plasmids were present.

For the WGS reads of the bacterial isolates, the whole INNUca v2.6 pipeline was run, including SPAdes assembly and polishing. 
Plasmids were detected as mentioned previously.

\subsubsection{Commercial-based approach}

The fastq files containing the reads were uploaded into CLC Genomics Workbench v10.1.1, using the following options: Illumina import, paired-reads, paired-end (forward-reverse) and minimum distance of 1 and a maximum distance of 1000 (default). 
The trimming was performed using the default settings, except the quality trimming score limit was set to 0.01 and we added a Trim adapter list containing Illumina adapters. 
The mapping was performed with the Map Reads to Reference tool, using the hg19 genome as reference. 
The default settings were used with the addition of the collect un-mapped reads option. The \textit{de novo} assembly tool was used for the assembly (even for the metagenomics reads) and, apart from the word size, which was changed to 29, all the settings were default. 
Two tools were used for the microbial identification, Taxonomic Profiling and Find Best Matches using K-mer Spectra (Microbial Genomics Module). 
In both, the bacterial and fungal databases were downloaded from NCBI RefSeq (with the Only Complete Genomes option turned off; minimum length 500,000 nucleotides) on 08-07-2017 (bacterial, 70,868 sequences) and 25-05-2017 (fungal, 377 sequences). 
The antimicrobial resistance genes were detected, based on the assembled contigs, using the Find Resistance tool (Microbial Genomics Module) and were initially only considered present when they were $\geq$ 70\% identical to the reference and $\geq$ 80\% of the sequence was covered. The analysis was also repeated using $\geq$ 40\% and $\geq$ 20\% of sequence coverage for comparison purposes. 
The database containing the antimicrobial resistance genes was downloaded directly to the software from ResFinder\footnote{\url{https://cge.cbs.dtu.dk/services/data.php}} (downloaded on 05-07-2017, 2156 sequences). The MLST was determined through the Identify MLST tool (Microbial Genomics Module), using all MLST schemes available at PubMLST (04-03-2017). 
The same database used for plasmid detection in Unix, was used for mapping the reads in CLC Genomics Workbench. Again, plasmids with less than 80\% of its length covered were excluded from the final results. 
For WGS reads we used the Trim Sequences tool and the assembly, antimicrobial resistance genes detection, and MLST determination were performed as before.

\subsubsection{Web-based approaches}

The fastq files containing the reads were uploaded into the BaseSpace\footnote{\url{https://basespace.illumina.com}} website. 
First, the raw forward and reverse fastq reads were subjected to FASTQ Toolkit for adapter/quality trimming and length filtering with standard settings and length filtering adjusted to a minimum of 100 and a maximum of 500. 
The trimmed reads were then used as input for all the following processes. 
The available microorganism identification apps Kraken v1.0.0, MetaPhlAn v1.0.0 and GENIUS v.1.1.0 were used with the standard settings/parameters.
SEAR was used to detect antimicrobial resistance genes, maintaining the standard settings except for the clustering stringency which was set to 0.98 and the annotation stringency was set to 40.
The SPAdes Genome Assembler v3.9.0 app was run with the standard parameters for multi cell data type. 
For metagenomic datatype settings, the running mode was set to only assembly and careful mode was disabled. 

The reads were uploaded into CosmosID\footnote{\url{https://app.cosmosid.com/login}} and Taxonomer\footnote{\url{https://www.taxonomer.com/}} \citep{flygare_taxonomer_2016} directly without any quality trimming. 
We used the Full Analysis mode in Taxonomer.

\subsubsection{wgMLST analyses}

Typing was done by MLST and wgMLST analyses obtained using Ridom SeqSphere+ v4.0.1. 
The genomic data (assembled contigs) obtained from SMg was compared to the data obtained through WGS.
Since no cg/wgMLST scheme was available for \textit{Escherichia coli}, \textit{Enterococcus faecalis}, \textit{Ochrobactrum intermedium} and \textit{Staphylococcus haemolyticus}, cgMLST and accessory genome schemes were constructed, using Ridom SeqSphere+ cgMLST Target Definer with the following parameters: a minimum length filter that removes all genes smaller than 50 bp; a start codon filter that discards all genes that contain no start codon at the beginning of the gene; a stop codon filter that discards all genes that contain no stop codon or more than one stop codon or that do not have the stop codon at the end of the gene; a homologous gene filter that discards all genes with fragments that occur in multiple copies within a genome (with identity of 90\% and >100 bp overlap); and a gene overlap filter that discards the shorter gene from the cgMLST scheme if the two genes affected overlap $>$4 bp. 
The remaining genes were then used in a pairwise comparison using BLAST version 2.2.12 (parameters used were word size 11, mismatch penalty −1, match reward 1, gap open costs 5, and gap extension costs 2). 
All genes of the reference genome that were common in all query genomes with a sequence identity of $\geq$ 90\% and 100\% overlap and, with the default parameter stop codon percentage filter turned on, formed the final cgMLST scheme. 
The combination of all alleles in each strain formed an allelic profile that was used to generate minimum spanning trees using the parameter “pairwise ignore missing values” during distance calculation \citep{ruppitsch_defining_2015}.

\subsubsection{Statistical analysis}

The sensitivity and positive predictive value of each taxonomic classification method were determined. 
Classical culture and MALDI-TOF identifications were considered as the gold standard. 
The true positives were considered when the same bacterial species were identified by culture/MALDI-TOF and the taxonomic classification method. 
The false positives were detected when bacterial species different from those identified by culture/MALDI-TOF, were identified by the taxonomic classification method. 
The false negatives were determined when the bacterial species identified by culture/MALDI-TOF were not identified by the taxonomic classification method.

\section{Results} \label{sec:ch2_results}

\subsection{Classical identification}

Nine body fluid samples and one tissue sample from 9 different patients were sequenced, including one sample from peritoneal fluid, five from pus (3 abscesses and 2 empyemas), two from synovial fluid of knees with prosthesis, one from sputum and one from a bone biopsy (Table \ref{tab:ch2_table_1}).
In total 15 different isolates obtained from the 10 samples were considered of possible clinical significance and were selected for species identification and antimicrobial susceptibility testing during routine work up of the samples (Table \ref{tab:ch2_table2},\ref{} and \ref{}).
In samples 2 and 3, only one colony-forming unit (CFU) of \textit{Escherichia coli} and \textit{Staphylococcus epidermidis}, respectively, was detected after 48 hours of incubation. 
In samples 2 and 5, the anaerobic cultures were mixed to such an extent, that no further characterization of the colonies was performed, and the results were reported as anaerobic mixed culture.

Antimicrobial susceptibility testing, revealed three isolates to be fully susceptible, while the others were resistant to at least one antimicrobial. 
Two isolates, one \textit{Staphylococcus haemolyticus} and one \textit{S. epidermidis} were oxacillin-resistant and positive in the cefoxitin test (Vitek 2).

There was fungal growth in 2 samples (1 and 5) that included two Candida species (one \textit{Candida glabrata} and one \textit{Candida albicans}). 
The different bacterial and fungal species identified in each sample are shown in Tables \ref{tab:ch2_table2}, \ref{} and \ref{}.


\begin{table}[]
\caption{Microorganisms identified by conventional methods, WGS and using shotgun metagenomics and the taxonomic classification methods in Unix.}
\label{tab:ch2_table2}
\resizebox{\textwidth}{!}{%
\begin{tabular}{@{}|l|l|l|l|lll|@{}}
\toprule
\multicolumn{1}{|c|}{\multirow{2}{*}{\textbf{Sample number}}} &
  \multicolumn{1}{c|}{\multirow{2}{*}{\textbf{Culture result (CFU)$^a$}}} &
  \multicolumn{1}{c|}{\multirow{2}{*}{\textbf{\begin{tabular}[c]{@{}c@{}}Conventional identification\\  (MALDI-TOF)\end{tabular}}}} &
  \multicolumn{1}{c|}{\multirow{2}{*}{\textbf{WGS-based identification}}} &
  \multicolumn{3}{c|}{\textbf{Shotgun metagenomics}} \\ \cmidrule(l){5-7} 
\multicolumn{1}{|c|}{} &
  \multicolumn{1}{c|}{} &
  \multicolumn{1}{c|}{} &
  \multicolumn{1}{c|}{} &
  \multicolumn{1}{c|}{\textbf{Kraken$^b$}} &
  \multicolumn{1}{c|}{\textbf{MIDAS$^c$}} &
  \multicolumn{1}{c|}{\textbf{MetaPhlAn$^c$}} \\ \midrule
\textbf{1} &
  \begin{tabular}[c]{@{}l@{}}10$^3$\\  10$^3$\\  10\end{tabular} &
  \begin{tabular}[c]{@{}l@{}}\textit{E. faecium}\\  \textit{S. haemolyticus}\\  \textit{C. glabrata}\end{tabular} &
  \begin{tabular}[c]{@{}l@{}}\textit{E. faecium}\\  \textit{S. haemolyticus}\\  -\end{tabular} &
  \multicolumn{1}{l|}{\begin{tabular}[c]{@{}l@{}}\textit{E. faecium} (34.6\%)\\  \textit{S. haemolyticus} (10.1\%)\\  -\end{tabular}} &
  \multicolumn{1}{l|}{\begin{tabular}[c]{@{}l@{}}\textit{E. faecium} (62.0\%)\\  \textit{S. haemolyticus} (28.0\%)\\  -\end{tabular}} &
  \begin{tabular}[c]{@{}l@{}}\textit{E. faecium} (66.6\%)\\  \textit{S. haemolyticys} (27.7\%)\\  -\end{tabular} \\ \midrule
\textbf{2} &
  \begin{tabular}[c]{@{}l@{}}10$^3$\\  1\\  Not determined\end{tabular} &
  \begin{tabular}[c]{@{}l@{}}\textit{E. avium}\\  \textit{E. coli}\\  Anaerobes\end{tabular} &
  \begin{tabular}[c]{@{}l@{}}-\#\\  -\#\\  -\#\end{tabular} &
  \multicolumn{1}{l|}{\begin{tabular}[c]{@{}l@{}}Not identified$^*$\\  Not identified$^*$\\  Several species (29.5\%)\end{tabular}} &
  \multicolumn{1}{l|}{\begin{tabular}[c]{@{}l@{}}Not identified$^*$\\  Not identified$^*$\\  Several species (100.0\%)\end{tabular}} &
  \begin{tabular}[c]{@{}l@{}}Not identified$^*$\\  Not identified$^*$\\  Several species (100.0\%)\end{tabular} \\ \midrule
\textbf{3} &
  1 &
  \textit{S. epidermidis} &
  -$^\#$ &
  \multicolumn{1}{l|}{S. aureus (0.2\%)} &
  \multicolumn{1}{l|}{Not identified$^*$} &
  Not identified$^*$ \\ \midrule
\textbf{4} &
  10$^3$ &
  \textit{S. aureus} &
  \textit{S. aureus} &
  \multicolumn{1}{l|}{\textit{S. aureus} (0.73\%)} &
  \multicolumn{1}{l|}{\textit{S. aureus }(100\%)} &
  \textit{S. aureus} (100\%) \\ \midrule
\textbf{5} &
  \begin{tabular}[c]{@{}l@{}}$\geq$ 10$^5$\\  $\geq$ 10$^5$\\  10$^3$\\  10$^3$\\  Not determined\\  10\end{tabular} &
  \begin{tabular}[c]{@{}l@{}}\textit{E. coli}\\  \textit{K. oxytoca}\\  \textit{S. anginosus}\\  \textit{E. faecalis}\\  Anaerobes\\  \textit{C. albicans}\end{tabular} &
  \begin{tabular}[c]{@{}l@{}}\textit{E. coli}\\  \textit{K. oxytoca}\\  -$^\#$\\  \textit{E. faecalis}\\  -$^\#$\\  -$^\#$\end{tabular} &
  \multicolumn{1}{l|}{\begin{tabular}[c]{@{}l@{}}\textit{E. coli} (9.7\%)\\  \textit{K. oxytoca} (0.5\%)\\  \textit{S. anginosus} (0.07\%)\\  \textit{E. faecalis} (0.3\%)\\  Several species (12.7\%)\\  -\end{tabular}} &
  \multicolumn{1}{l|}{\begin{tabular}[c]{@{}l@{}}\textit{E. coli} (6.5\%)\\  \textit{K. oxytoca} (0.3\%)\\  \textit{S. anginosus} (0.01\%)\\  \textit{E. faecalis} (0.9\%)\\  Several species (96.7\%)\\  -\end{tabular}} &
  \begin{tabular}[c]{@{}l@{}}\textit{E. coli} (8.5\%)\\  \textit{K. oxytoca} (0.3\%)\\  \textit{Streptococcus spp.} (0.09\%)\\  \textit{E. faecalis} (0.7\%)\\  Several species (90.4\%)\\  -\end{tabular} \\ \midrule
\textbf{6} &
  10$^3$ &
  \textit{E. faecium} &
  \textit{E. faecium} &
  \multicolumn{1}{l|}{\textit{E. faecium} (0.77\%)} &
  \multicolumn{1}{l|}{Not identified$^*$} &
  Not identified$^*$ \\ \midrule
\textbf{7} &
  10$^2$ &
  \textit{S. aureus} &
  -$^\#$ &
  \multicolumn{1}{l|}{\textit{S. aureus} (82.9\%)} &
  \multicolumn{1}{l|}{\textit{S. aureus} (100\%)} &
  \textit{S. aureus} (100\%) \\ \midrule
\textbf{8} &
  10$^3$ &
  \textit{O. intermedium} &
  \textit{O. intermedium} &
  \multicolumn{1}{l|}{\textit{O. anthropi} (21.3\%)} &
  \multicolumn{1}{l|}{\textit{O. intermedium} (99.4\%)} &
  \textit{O. intermedium} (99.1\%) \\ \midrule
\textbf{9} &
  10$^3$ &
  \textit{S. aureus} &
  \textit{S. aureus} &
  \multicolumn{1}{l|}{\textit{S. aureus} (22.9\%)} &
  \multicolumn{1}{l|}{\textit{S. aureus} (100\%)} &
  \textit{S. aureus} (100\%) \\ \midrule
\textbf{10} &
  10$^3$ &
  \textit{S. marcescens} &
  -\# &
  \multicolumn{1}{l|}{\textit{S. marcescens} (64.7\%)} &
  \multicolumn{1}{l|}{\textit{S. marcescens} (99.1\%)} &
  \textit{S. marcescens} (100\%) \\ \bottomrule
\end{tabular}%
}
\tiny
\item $^a$The number of colonies of a given species was estimated from the number of colonies with the same morphology on the same plate 
\item $^b$The relative abundance is calculated using total number of reads as denominator
\item $^c$The relative abundance is calculated with the total number of classified reads as denominator
\item $^d$miniKraken database was used
\item $^\#$Although there was a laboratory identification, no isolates were available for WGS
\item $^*$No reads matched that specific pathogen, not even at the genus level
\end{table}

\begin{table}[]
\caption{Microorganisms identified by conventional methods, WGS and using shotgun metagenomics and the taxonomic classification methods in CLC Genomics Workbench.}
\label{tab:ch2_table3}
\resizebox{\textwidth}{!}{%
\begin{tabular}{@{}|l|l|l|l|ll|@{}}
\toprule
\multicolumn{1}{|c|}{\multirow{2}{*}{\textbf{Sample number}}} &
  \multicolumn{1}{c|}{\multirow{2}{*}{\textbf{Culture result (CFU)$^a$}}} &
  \multicolumn{1}{c|}{\multirow{2}{*}{\textbf{\begin{tabular}[c]{@{}c@{}}Conventional identification\\  (MALDI-TOF)\end{tabular}}}} &
  \multicolumn{1}{c|}{\multirow{2}{*}{\textbf{WGS-based identification}}} &
  \multicolumn{2}{c|}{\textbf{Shotgun metagenomics}} \\ \cmidrule(l){5-6} 
\multicolumn{1}{|c|}{} &
  \multicolumn{1}{c|}{} &
  \multicolumn{1}{c|}{} &
  \multicolumn{1}{c|}{} &
  \multicolumn{1}{l|}{\textbf{Taxonomic Profiling (CLC)$^b$}} &
  \textbf{Best match with K-mer spectra (CLC)$^c$} \\ \midrule
\textbf{1} &
  \begin{tabular}[c]{@{}l@{}}103\\  10$^3$\\  10\end{tabular} &
  \begin{tabular}[c]{@{}l@{}}\textit{E. faecium}\\  \textit{S. haemolyticus}\\  \textit{C. glabrata}\end{tabular} &
  \begin{tabular}[c]{@{}l@{}}\textit{E. faecium}\\  \textit{S. haemolyticus}\\  -\end{tabular} &
  \multicolumn{1}{l|}{\begin{tabular}[c]{@{}l@{}}\textit{E. faecium} (71\%)\\  \textit{S. haemolyticus} (24\%)\\  \textit{C. glabrata} (100\%)\end{tabular}} &
  \begin{tabular}[c]{@{}l@{}}\textit{E. faecium} (41.4\%)\\  \textit{S. haemolyticus} (13.8\%)\\  \textit{C. glabrata} (0.5\%)\end{tabular} \\ \midrule
\textbf{2} &
  \begin{tabular}[c]{@{}l@{}}10$^3$\\  1\\  Not determined\end{tabular} &
  \begin{tabular}[c]{@{}l@{}}\textit{E. avium}\\  \textit{E. coli}\\  Anaerobes\end{tabular} &
  \begin{tabular}[c]{@{}l@{}}-$^\#$\\  -$^\#$\\  -$^\#$\end{tabular} &
  \multicolumn{1}{l|}{\begin{tabular}[c]{@{}l@{}}Not identified$^*$\\  Not identified$^*$\\  Several species (97\%)\end{tabular}} &
  \begin{tabular}[c]{@{}l@{}}Not identified$^*$\\  Not identified$^*$\\  Several species (13.2\%)\end{tabular} \\ \midrule
\textbf{3} &
  1 &
  \textit{S. epidermidis} &
  -\# &
  \multicolumn{1}{l|}{Not identified$^*$} &
  \textit{S. aureus} (4\%) \\ \midrule
\textbf{4} &
  10$^3$ &
  \textit{S. aureus} &
  \textit{S. aureus} &
  \multicolumn{1}{l|}{Not identified$^*$} &
  \textit{S. aureus} (9.7\%) \\ \midrule
\textbf{5} &
  \begin{tabular}[c]{@{}l@{}}$\geq$ 10$^5$\\  $\geq$ 10$^5$\\  10$^3$\\  10$^3$\\  Not determined\\  10\end{tabular} &
  \begin{tabular}[c]{@{}l@{}}\textit{E. coli}\\  \textit{K. oxytoca}\\  \textit{S. anginosus}\\  \textit{E. faecalis}\\  Anaerobes\\  \textit{C. albicans}\end{tabular} &
  \begin{tabular}[c]{@{}l@{}}\textit{E. coli}\\  \textit{K. oxytoca}\\  -$^\#$\\  E. faecalis\\  -$^\#$\\  -$^\#$\end{tabular} &
  \multicolumn{1}{l|}{\begin{tabular}[c]{@{}l@{}}\textit{E. coli} (25\%)\\  \textit{K. michiganensis} (0.3\%)\\  Not identified$^*$\\  \textit{E. faecalis} (2\%)\\  Several species (70.0\%)\\  Not identified$^*$\end{tabular}} &
  \begin{tabular}[c]{@{}l@{}}\textit{E. coli} (11.5\%)\\  Not identified$^*$\\  Not identified$^*$\\  \textit{E. faecalis} (0.6\%)\\  Not identified$^*$\\  \textit{C. albicans} (\textless{}0.05\%)\end{tabular} \\ \midrule
\textbf{6} &
  10$^3$ &
  \textit{E. faecium} &
  \textit{E. faecium} &
  \multicolumn{1}{l|}{Not identified$^*$} &
  \textit{E. faecium} (4.0\%) \\ \midrule
\textbf{7} &
  10$^2$ &
  \textit{S. aureus} &
  -$^\#$&
  \multicolumn{1}{l|}{\textit{S. aureus} (100\%)} &
  \textit{S. aureus} (95.5\%) \\ \midrule
\textbf{8} &
  10$^3$ &
  \textit{O. intermedium} &
  \textit{O. intermedium} &
  \multicolumn{1}{l|}{\textit{O. intermedium }(86.0\%)} &
  \textit{O. intermedium} (91.2\%) \\ \midrule
\textbf{9} &
  10$^3$ &
  \textit{S. aureus} &
  \textit{S. aureus} &
  \multicolumn{1}{l|}{\textit{S. aureus} (100\%)} &
  \textit{S. aureus} (81.2\%) \\ \midrule
\textbf{10} &
  10$^3$ &
  \textit{S. marcescens} &
  -$^\#$ &
  \multicolumn{1}{l|}{\textit{S. marscescens} (100\%)} &
  \textit{S. marcescens} (79.7\%) \\ \bottomrule
\end{tabular}%
}
\tiny
\item $^a$The number of colonies of a given species was estimated from the number of colonies with the same morphology on the same plate 
\item $^b$The relative abundance is calculated using total number of reads as denominator
\item $^c$The relative abundance is calculated with the total number of classified reads as denominator
\item $^d$miniKraken database was used
\item $^\#$Although there was a laboratory identification, no isolates were available for WGS
\item $^*$No reads matched that specific pathogen, not even at the genus level
\end{table}
