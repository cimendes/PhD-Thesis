\mbox{}\\
\vspace{8cm}

This chapter is a reproduction of the following submitted manuscript for publication in GigaScience:

C. I. Mendes, P. Vila-Cerqueira, Y. Motro, J. Moran-Gilad, J. A. Carriço, M. Ramirez. LMAS: Last Metagenomic Assembler Standing

The supplementary information referred throughout the text can be consulted in this chapter before the section of references.

Short-read \ac{SMg} can offer comprehensive microbial detection and characterisation of complex clinical samples. The \textit{de novo} assembly of raw sequence data is key in metagenomic analysis, yielding longer sequences that offer contextual information and afford a more complete picture of the microbial community. The assembly process is the bedrock and may constitute a major bottleneck in obtaining trustworthy, reproducible results.

In this chapter, we present LMAS, an automated workflow developed as a flexible platform to allow users to evaluate traditional and metagenomic dedicated prokaryotic de novo assembly software performance given known standard communities. Its implementation in Nextflow ensures the transparency and reproducibility of the results obtained and the use of Docker containers provides further flexibility. The results are presented in an interactive HTML report where global and reference specific performance metrics can be explored. Currently, 12 assemblers still being maintained are implemented in LMAS, with the possibility of expansion as novel algorithms are developed.

To showcase LMAS we used a test dataset of eight bacterial genomes and four plasmids of the ZymoBIOMICS Microbial Community Standards with linear and logarithmic distribution,  and found that k-mer De Bruijn graph assemblers outperformed the alternative approaches but came with a greater computational cost. Furthermore, assemblers branded as metagenomic specific did not consistently outperform other genomic assemblers in metagenomic samples. Some assemblers still in use, such as ABySS, BCALM2, MetaHipmer2, minia and VelvetOptimiser,  showed significant performance problems and their usability may be limited, particularly when assembling complex samples. 

The performance of each assembler varied depending on the species of interest and its abundance in the sample, with less abundant species presenting a significant challenge for all assemblers. No assembler stood out as an undisputed all-purpose choice for short-read metagenomic prokaryote genome assembly, highlighting that efforts are still needed to further improve metagenomic assembly performance. Our results also suggest that sample complexity and a particular interest in some sample components may affect assembler choice. Using LMAS could help users in their choice of assembler for their specific purpose.  As such, we believe that this manuscript is appropriate for publication in Microbiome as a Software article. 

My contribution to this publication included the design, implementation and optimisation of the LMAS the workflow, including the creation of the Docker containers for all dependencies. I performed the data analysis and comparison of assemblers included in LMAS with ZymoBIOMICS Microbial Community Standards, both evenly and logarithmically distributed samples. Additionally, I've also wrote the manuscript.

\cleardoublepage 

\begin{center}
\large
\textbf{LMAS: Last Metagenomic Assembler Standing}
\end{center}

Catarina I Mendes$^1,*$, 
P Vila-Cerqueira$^1*$,
Y Motro$^2$,
J. Moran-Gilad$^2$,
João A Carriço$^1$
Mário Ramirez$^1$, 


$^1$Instituto de Microbiologia, Instituto de Medicina Molecular, Faculdade de Medicina, Universidade de Lisboa, Lisboa, Portugal 

$^2$Faculty of Health Sciences, Ben-Gurion University of the Negev, Beer-Sheva, Israel


