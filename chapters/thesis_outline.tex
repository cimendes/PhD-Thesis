The work described in the present thesis intended to evaluate the use of bioinformatics methods for the analysis of metagenomic data to allow the rapid identification, virulence analysis and antimicrobial susceptibility prediction of pathogens with clinical relevance. Ultimately, the applicability of metagenomic methods is to be evaluated in a clinical setting as an alternative to current golden standards. Given the dependence of these methodologies on bioinformatics post-processing of the raw data obtained, the major applications and pitfalls of metagenomics are yet to be identified. 

The thesis comprises 10 chapters, organised as follows:

In \textbf{Chapter 1} the issues addressed throughout the thesis are put into context, highlighting the current impact of genomics in clinical microbiology, both as a diagnostic or a surveillance tool. The entire process in clinical microbiology for bacterial and viral infections is showcased through its different approaches over time: classical biochemical and molecular methods, whole-genome sequencing, and sequencing through metagenomics, both metataxonomics and shotgun, with a focus on the computational requirements necessary. This chapter elaborates on the evolution of whole-genome sequencing to metagenomic approaches, introducing the possibility of the identification and characterisation of a potential pathogen without the need for a priori knowledge of the causative agent of disease. The importance of bioinformatics analysis of these data was underlined, showcasing its complexity and the major pitfalls, such as reproducibility and transparency of the analysis methods.   

\textbf{Chapter 2} consists of the application of the shotgun metagenomics approach to nine body fluid samples and one tissue sample from patients at the University Medical Center Groningen (UMCG) as to compare against current golden standards practises in the diagnosis of disease. In this study, the accuracy and reliability of the bioinformatics analyses were evaluated and compared against the results obtained from traditional culture methods. Our aim was to evaluate the applicability of shotgun metagenomics in a routine diagnostic setting, and not only in cases where traditional methods fail to provide an answer. Most pathogens identified by culture were also identified by metagenomics. Substantial differences were noted between the taxonomic classification tools, highlighting the potential and limitations of shotgun metagenomics as a diagnostic tool. The fact that, when applying shotgun metagenomics to diagnostics, the results are highly dependent on the tools, and especially the database that was chosen for the analysis greatly impacts its applicability in a clinical setting. This chapter is included in the following publication:\textit{ N. Couto, L. Schuele, E.C. Raangs, M. P. Machado, \underline{C. I. Mendes}, T. F. Jesus, M. Chlebowicz,  S. Rosema, M. Ramirez, J. A. Carriço, I. B. Autenrieth, A. W. Friedrich, S. Peter and J. W. Rossen. Critical steps in clinical shotgun metagenomics for the concomitant detection and typing of microbial pathogens. Sci Rep 8, 13767 (2018). DOI: \url{https://doi.org/10.1038/s41598-018-31873-w}}

\textbf{Chapter 3} describes the application of both second and third-generation sequencing technologies, also known as next-generation and long-read sequencing, to tap-water samples collected at the University Medical Center Groningen. Our aim was to evaluate the applicability of shotgun metagenomics, but this time in a surveillance setting. Building on the findings from Chapter 2, a hybrid assembly approach was used to increase resolution power. In this sample a new variant of a colistin resistance (mcr) determinants was detected, named mcr-5.4, and through hybrid assembly leveraging both short and long-read sequences, its context was determined, albeit with questionable success. This chapter is included in the following publication: \textit{G. Fleres, N. Couto, L. Schuele, M. A. Chlebowicz, \underline{C. I. Mendes}, L. W. M. van der Sluis, J. W. A. Rossen, A. W Friedrich, S. García-Cobos, Detection of a novel mcr-5.4 gene variant in hospital tap water by shotgun metagenomic sequencing, Journal of Antimicrobial Chemotherapy, Volume 74, Issue 12, December 2019, Pages 3626–3628.  DOI: \url{https://doi.org/10.1093/jac/dkz363.}}

With the lessons learnt in Chapters 2 and 3, we developed in \textbf{Chapter 4} DEN-IM, a one-stop, user-friendly, containerised and reproducible workflow for the analysis of Dengue virus short-read sequencing data from both amplicon and shotgun metagenomics approaches. This takes into particular consideration the dependency on software and database versions used in the metagenomic bioinformatics downstream analysis in the results obtained. Dengue virus represents a public health threat and economic burden in affected countries, with the risk of exposure, increasing, not only driven by travel to endemic regions but also due to the broader dissemination of the mosquito vector, making the burden of dengue very significant. This makes it a particularly relevant target organism for the development of a straightforward workflow for both the identification and characterization of the virus. DEN-IM was designed to perform a comprehensive analysis in order to generate either de novo assemblies or consensus of full viral coding sequences and to identify their serotype and genotype, including the identification of co-infection cases whose prevalence is increasingly found in highly endemic areas. It was developed in Nextflow,  a simple and scalable workflow management system. All tools and dependencies are provided in Docker containerised images. All these steps ensure reproducibility and transparency of the bioinformatic process. This chapter is included in the following publication: \textit{\underline{C. I. Mendes}\footnote[1]{These authors contributed equally to this work.}, E. Lizarazo$^*$ , M. P. Machado, D. N. Silva, A. Tami, M. Ramirez, N. Couto, J. W. A. Rossen, J. A. Carriço, DEN-IM: dengue virus genotyping from amplicon and shotgun metagenomic sequencing. Microbial Genomics, Volume 6, Issue 3, March 2020. DOI: \url{https://doi.org/10.1099/mgen.0.000328.}}

A key process in metagenomic data analysis is the de novo assembly of raw sequence data since it allows recovering contigs representing the replicons present in the sample, be it genomes, plasmids, or bacteriophages, from a pool of mixed raw reads. \textbf{Chapter 5} employs the same core principles as in Chapter 4, describing a one-stop, user-friendly, containerised, and reproducible workflow, named LMAS, to assess the performance of de novo assembly algorithms for the assembly of second-generation metagenomic sequencing data. The LMAS workflow, which allows users to evaluate performance given a known standard community was implemented in Nextflow, ensuring the transparency and reproducibility of the results obtained. Similarly to Chapter 4, the use of Docker containers provides additional flexibility. The results are presented in an interactive HTML report where global and reference specific performance metrics can be explored. Currently, 12 de novo assemblers are implemented in LMAS, with the possibility of expansion as novel algorithms are developed. 

Despite the advantages of reproducible, containerised workflow, Chapters 4 and 5 still do not guarantee the interoperability of results obtained from various sources. Chapter 5 highlighted the impact that the tool choice can have on downstream results when working with metagenomic data, therefore, and due to the lack of standardisation, it is pivotal that results from various tools can be compared for their applicability in the clinic. With a focus on antimicrobial resistance, \textbf{Chapter 6} presents a standardised output specification for the bioinformatic detection of antimicrobial resistance directly from genomes or metagenomes. This addresses the problem of combining the outputs of disparate antimicrobial resistance gene detection tools into a single unified format, implemented into a python package and command-line utility hAMRonization. As the detection of antimicrobial resistance directly from genomic or metagenomic data has become a standard procedure in public health, with hAMRonization allowing for the comparison of results within bioinformatics workflows, as these tools, although implementing similar principles, differ in supported inputs, search algorithms, parameterisation, and underlying reference databases. 

\textbf{Chapter 7} presents a direct application of a standardised specification, such as the one presented in Chapter 6. For this purpose, a SARS-CoV-2 contextual data specification package based on harmonisable, publicly available community standards was developed and implemented through a collection template, as well as a variety of protocols and tools to support both the harmonisation and submission of sequence data and contextual information to public biorepositories. In addition to the reproducibility and interoperability of data and software, transparency is also a keystone in the use of bioinformatics methods for the analysis of metagenomic data. This chapter is included in the following publication: \textit{E. J. Griffiths, R. E. Timme, \underline{C. I. Mendes}, A. J. Page, N. Alikhan, D. Fornika, F. Maguire, J. Campos, D. Park, I. B. Olawoye, P. E. Oluniyi, D. Anderson, A. Christoffels, A. G. da Silva, R. Cameron, D. Dooley, L. S. Katz, A. Black, I. Karsch-Mizrachi, T. Barrett, A. Johnston, T. R. Connor, S. M. Nicholls, A. A. Witney, G. H. Tyson, S. H. Tausch, A. R. Raphenya, B. Alcock, D. M. Aanensen, E. Hodcroft, W. W. L. Hsiao, A. T. R. Vasconcelos, D. R. MacCannell on behalf of the Public Health Alliance for Genomic Epidemiology (PHA4GE) consortium, Future-proofing and maximizing the utility of metadata: The PHA4GE SARS-CoV-2 contextual data specification package. GigaScience, Volume 11, 2022, giac003. DOI: \url{https://doi.org/10.1093/gigascience/giac003.}} 

\textbf{Chapter 8} showcases an effort to raise standards on the development and distribution of code for bioinformatic analysis. For this, seven recommendations are presented that help researchers implement software testing in microbial bioinformatics. We propose collaborative software testing as an opportunity to continuously engage software users, developers, and students to unify scientific work across domains. As automated software testing remains underused in scientific software, our set of recommendations not only ensures that appropriate effort can be invested in producing high quality and robust software, but also increases engagement in its sustainability. This chapter is included in the following publication: \textit{B. C. L. van der Putten$^*$, \underline{C. I. Mendes}\footnote[1]{These authors contributed equally to this work.}, B. M. Talbot, J. de Korne-Elenbaas, R. Mamede, P. Vila-Cerqueira, L. P. Coelho, C. A. Gulvik, L. S. Katz, The Asm Ngs Hackathon Participants, Software testing in microbial bioinformatics: a call to action. Microbial Genomics, Volume 8, Issue 3. DOI: \url{https://doi.org/10.1099/mgen.0.000790.}}

\textbf{Chapter 9} corresponds to the general discussion. This chapter provides a summary of the main results obtained in this thesis and its integrated discussion. It is divided into two main sections: the current limitations to the application of metagenomics in clinical microbiology; and the better standards required for metagenomics to become a standard microbiological method, with a clearly defined role in both diagnosis and surveillance. For the first, three major limitations were identified, starting with the limitations inherent to the sequencing technology itself, followed by the unbiased nature of metagenomics, very sensitive to host and/or environmental contamination, and ending in the limitation of the bioinformatic analysis itself, where no standard procedure is \textit{de facto} accepted. Several steps are required to improve the standards in metagenomics before its routine application. The need for proper benchmarks, with the use of well-characterised communities, is paramount for protocol validation. Likewise, the adoption of reproducible and auditable workflows, relying upon well-established software is just as important as wet-lab procedures, with just as much influence on the validity of the results obtained. The use of intuitive and responsive reports will allow clinical and research personnel without the technical know-how to infer knowledge from the complex analysis required for the application of metagenomics. The application of data standards, with controlled vocabulary, will also contribute to crossing the data-to-informative-report bridge. Finally, the community needs to be engaged in adopting these practices, with crowdsourcing being a viable option for the dissemination of better standards worldwide. 


\textbf{Chapter 10} Contains the main conclusions driven from this work and also perspectives for future work. 