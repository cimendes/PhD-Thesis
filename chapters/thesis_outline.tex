The work described in the present thesis intended to XXXXX.

The thesis comprises X chapters, organised as follows:

\textbf{Chapter 1} corresponds to the general introduction that highlights the impact of genomics in clinical microbiology, both in its diagnostic and surveillance prongs. This chapter further expands on this topic by expanding the concept of whole genome sequencing to metagenomics, both metataxonomics and shotgun, introducing the possibility of the identification and characterisation of a potential pathogen without the need for \textit{a priori} knowledge. In this chapter, the importance of bioinformatic analysis of these data is highlighted, showcasing its complexity and the major pitfalls, such as reproducibility and transparency of the results obtained. succinctly, the entire process in clinical microbiology for bacterial and viral infections is showcased through its different approaches: classical biochemical and molecular methods, whole genome sequencing, and sequencing through metagenomics, with a focus on the computational requirements necessary. 

\textbf{Chapter 2} consists in the application of shotgun metagenomics approaches to  nine body fluid samples and one tissue sample from patients at the University Medical Center Groningen (UMCG). In this study, the accuracy and reliability of the bioinformatics analyses was evaluated and compared against the results obtained from traditional culture methods. Most pathogens identified by culture were also identified by metagenomics, but substantial differences were noted between the taxonomic classification tools, highlighting the potential and limitations of shotgun metagenomics as a diagnostic tool, but also the fact that, when applying shotgun metagenomics to diagnostics, the results are highly dependent on the tools, and especially the database chosen for the analysis.

\textbf{Chapter 3} consists in ...

\textbf{Chapter 4}

\textbf{Chapter 5}

\textbf{Chapter 6}

\textbf{Chapter 7}

\textbf{Chapter 8} corresponds to the general discussion. In this chapter is provided a
summary of the main results obtained in this thesis and its integrated discussion. 

\textbf{Chapter 9} Contains the main conclusions driven from this work. It also includes perspectives for future work. 