\begin{table}[]
\centering
\caption{Resources that form the PHA4GE SARS-CoV-2 contextual data specification package}
\label{tab:ch7_table2}
\resizebox{\textwidth}{!}{%
\begin{tabular}{@{}ll@{}}
\toprule
Resource$^1$ &
  Description \\
 \midrule
\begin{tabular}[c]{@{}l@{}}Collection template and \\ controlled vocabulary pick lists\end{tabular} &
  \begin{tabular}[c]{@{}l@{}}Spreadsheet-based collection form containing different fields \\ (identifiers and accessions, sample collection and processing,\\ host information, host exposure, vaccination and reinfection \\ information, lineage and variant information, sequencing, \\ bioinformatics and QC metrics, diagnostic testing information, \\ author acknowledgements). Fields are colour-coded to indicate \\ required, recommended or optional status. Many fields offer \\ pick lists of controlled vocabulary. Vocabulary lists are also \\ available in a separate tab.\end{tabular} \\  \midrule
Reference guides &
  \begin{tabular}[c]{@{}l@{}}Field and term definitions, guidance, and examples are \\ provided as separate tabs in the collection template .xlsx file \\ (see Term Reference Guide and Field Reference Guide).\end{tabular} \\  \midrule
Curation protocol on protocols.io &
  \begin{tabular}[c]{@{}l@{}}Step-by-step instructions for using the collection template are \\ provided in a standard operating procedure (SOP). Ethical, \\ practical, and privacy considerations are also discussed. \\ Examples and instructions for structuring sample descriptions \\ as well as sourcing additional standardized terms (outside \\ those provided in pick lists) are also discussed.\end{tabular} \\  \midrule
\begin{tabular}[c]{@{}l@{}}Mapping file of PHA4GE \\ fields to metadata standards\end{tabular} &
  \begin{tabular}[c]{@{}l@{}}PHA4GE fields are mapped to existing metadata standards \\ such as the Sample Application Standard, MIxS 5.0, and the \\ MIGS Virus Host-associated attribute package. Mappings are \\ available in the Reference guide tab. Mappings highlight which \\ fields of these standards are considered useful for SARS-CoV-2  \\ public health surveillance and investigations, and which fields \\ are considered out of scope.\end{tabular} \\  \midrule
\begin{tabular}[c]{@{}l@{}}Mapping of PHA4GE fields \\ to WHO metadata recommendations\end{tabular} &
  \begin{tabular}[c]{@{}l@{}}PHA4GE  fields are mapped to corresponding contextual data \\ elements recommended by the World Health Organization.\end{tabular} \\  \midrule
\begin{tabular}[c]{@{}l@{}}Mapping file of PHA4GE fields \\ to EMBL-EBI, NCBI and GISAID \\ submission requirements\end{tabular} &
  \begin{tabular}[c]{@{}l@{}}Many PHA4GE fields have been sourced from public repository \\ submission requirements. The different repositories have different \\ requirements and field names. Repository submission fields have \\ been mapped to PHA4GE fields to demonstrate equivalencies and \\ divergences.\end{tabular} \\  \midrule
\begin{tabular}[c]{@{}l@{}}Data submission protocol \\ (NCBI) on protocols.io\end{tabular} &
  \begin{tabular}[c]{@{}l@{}}The SARS-CoV-2 submission protocol for NCBI provides step-by-step \\ instructions and recommendations aimed at improving interoperability \\ and consistency of submitted data.\end{tabular} \\  \midrule
\begin{tabular}[c]{@{}l@{}}Data submission protocol \\ (EMBL-EBI) on protocols.io\end{tabular} &
  \begin{tabular}[c]{@{}l@{}}The SARS-CoV-2 submission protocol for ENA provides step-by-step \\ instructions and recommendations aimed at improving interoperability \\ and consistency of submitted data.\end{tabular} \\  \midrule
\begin{tabular}[c]{@{}l@{}}Data submission protocol \\ (GISAID) on protocols.io\end{tabular} &
  \begin{tabular}[c]{@{}l@{}}The SARS-CoV-2 submission protocol for GISAID provides step-by-step \\ instructions and recommendations aimed at improving interoperability \\ and consistency of submitted data.\end{tabular} \\  \midrule
\begin{tabular}[c]{@{}l@{}}JSON structure of \\ PHA4GE specification\end{tabular} &
  \begin{tabular}[c]{@{}l@{}}A JSON structure of the PHA4GE specification has been provided for easier \\ integration into software applications.\end{tabular} \\  \midrule
\begin{tabular}[c]{@{}l@{}}PHA4GE template in \\ the DataHarmonizer\end{tabular} &
  \begin{tabular}[c]{@{}l@{}}Javascript application enabling standardized data entry, validation and \\ export of contextual data as submission-ready forms for GISAID and \\ NCBI. The SOP for using the software can be found at \\ https://github.com/Public-Health-Bioinformatics/\\ DataHarmonizer/wiki/PHA4GE-SARS-CoV-2-Template\end{tabular}  
\end{tabular}%
}
\small
\item $^1$ There are a number of resources that form the PHA4GE SARS-CoV-2 contextual data specification package which are described in the table. The package has been compiled to support user implementation and data sharing, with integration into workflows and new software applications in mind.
\end{table}